%
% ^^A Configuring DocStrip [[[
% \iffalse
%<*driver>
\input mydocstrip
\mygenerate[output-files/syntax]{
  dot.vim,
  python.vim,
  tex.vim,
}
\myendbatchfile
\mydriver[
  show-notes,
]
%</driver>
% \fi
% ^^A ]]] End of Configuring DocStrip
%
% \mytitle{Making own Vim syntax files}
%
% \begin{concealableitemize}^^A [[[
% \1 The origin of `making own Vim syntax files'
%   \2 \doublequotes{MAKING YOUR OWN SYNTAX FILES}の記述\vimhelp{mysyntaxfile}
%   \2 \doublequotes{syntax}の由来は\vimhelp{mysyntaxfile}のフォルダ名\verb|syntax|
% \end{concealableitemize}^^A ]]]
%
% \begin{abstract}^^A [[[
% Vim Syntax Files (\verb|tex.vim|, ...)のメモ
% \end{abstract}^^A ]]]
%
% \mytableofcontents
%
% \section{Introduction}^^A [[[
% \verb|syn|の文法は\verb|:h syn-arguments|などを参照.
%
% 関数名は大文字始まり(\verb|TexNewMathZone(), SuperSub(), ...|),
% ***は小文字始まり(\verb|texMathSymbol, texRefZone, ...|).
%
% \subsection{注意}^^A [[[
% 既存の関数を使用したい場合,\verb|after|フォルダの下にsyntaxファイルを置くこと.
% 例えば,\verb|tex.vim|を\verb|~/.cache/dein/.cache/init.vim/.dein/syntax/|に置く場合,
% \verb|TexNewMathZone|関数が認識されない(未定義となりエラー).
% ^^A ]]] End of subsection `注意'.
%
% ^^A ]]] End of section `Introduction'.
%
% \section{Regular Expression}^^A [[[
% タイトルは\href{https://en.wikipedia.org/wiki/Regular_expression}{Wiki}より.
%
% 正規表現の\href{https://regexr.com/}{テストサイト}は有益.
%
% \href{http://www-creators.com/archives/1804#Vim}{正規表現:最短一致でマッチさせる表現}
%
% \subsection{***}^^A [[[
% この節の説明は\href{https://regexr.com/}{ここ}に記載されているものである.
%
% \subsubsection{\texcommand{}\texcommand{}}^^A [[[
% \textsf{Escaped character}. Matches a ``\verb|\|" character.
% ^^A ]]] End of subsubsection `\\'.
%
% \subsubsection{\texttt{\{},\texttt{\}}}^^A [[[
% \textsf{Character}. Matches a ``\verb|{|,\verb|}|" character.
% ^^A ]]] End of subsubsection `{, }'.
%
% \subsubsection{\texttt{.}}^^A [[[
% \textsf{Dot}. Matches any character except line breaks.
% ^^A ]]] End of subsubsection `.'.
%
%
%
%
%
% ^^A ]]] End of subsection `***'.
%
% ^^A ]]] End of section `Regular Expression'.
%
% \part{\myverb{dot.vim}}^^A [[[
% \iffalse
%<*dot.vim>
% \fi
%
% \section{\texttt{.dtx} file}^^A [[[
% \begin{myitemize}
% \1 下記コードは次のファイルを参考にした:\href{run:/usr/share/nvim/runtime/syntax/dot.vim}{\texttt{dot.vim}} \myline{31}
% \1 
% \end{myitemize}
%
% \begin{vimcode}
syn match   dotComment "^%.*$" contains=dotComment,dotTodo
% \end{vimcode}
% ^^A ]]] End of section `.dtx file'.
%
% \iffalse
%</dot.vim>
% \fi
% ^^A ]]] End of part `dot.vim'.
%
% \part{\myverb{python.vim}}^^A [[[
% \iffalse
%<*python.vim>
% \fi
%
% \section{\texttt{.dtx} file}^^A [[[
% \begin{myitemize}
% \1 \verb|.dtx|ファイル内でPythonコードを書いた場合,\verb|:set syntax=python|して,
% 下記コマンドを実行するとtex記述をコメントとしてハイライトできる
% \1 下記コードは次のファイルを参考にした:\href{run:/usr/share/nvim/runtime/syntax/python.vim}{\texttt{python.vim}} \myline{120}
% \1 
% \end{myitemize}
%
% \begin{vimcode}
syn match   pythonComment "^%.*$" contains=pythonTodo,@Spell
% \end{vimcode}
% ^^A ]]] End of section `.dtx file'.
%
% \iffalse
%</python.vim>
% \fi
% ^^A ]]] End of part `python.vim'.
%
% \part{\myverb{tex.vim}}^^A [[[
% \iffalse
%<*tex.vim>
% \fi
%
% \href{http://www.drchip.org/astronaut/vim/index.html#LATEXPKGS}{ここ}にamsmath用のsyntax fileがある
% (\verb|:h tex-morecomands|).使用するか要検討.
%
% \section{Syntax highlighting}^^A [[[
% タイトルはテキトーにつけた.
%
% \subsubsection{molokaiの色メモ}^^A [[[
% molokaiに用いられている色
%
% \href{https://jonasjacek.github.io/colors/}{256 COLORS - CHEAT SHEET}
%
% \begin{center}
% \begin{tabular}{ll}
%	color	& number	\\ \hline
%	black	& 0		\\^^A 確認済み(Search)
%	green	& 118	\\^^A 確認済み(Function)
%	orange	& 208	\\^^A 確認済み(CursorLineNr)
%	purple	& 135	\\^^A 確認済み(Number)
%	red		& 161	\\^^A 確認済み(keyword)
%	white	& 	\\
% \end{tabular}
% \end{center}
% ^^A ]]] End of subsubsection `***'.
%
% \subsection{Defining Conceal}^^A [[[
% \href{https://github.com/pangloss/vim-javascript/issues/151}{参照先}
% \begin{vimcode}
hi Conceal ctermfg=0 ctermbg=135 cterm=bold,underline
% \end{vimcode}
% ^^A ]]] End of subsection `***'.
%
% \subsection{Defining ***}^^A [[[
% \verb|\begin|などのフォントを変更.
% \begin{vimcode}
hi Statement ctermfg=161 cterm=bold
% \end{vimcode}
% ^^A ]]] End of subsection `***'.
%
% \subsection{Defining ***}^^A [[[
% \verb|\begin{foo}|の\verb|foo|の部分や\verb|\verb|のフォントを変更.
% \begin{vimcode}
hi PreCondit ctermfg=118 cterm=bold
% \end{vimcode}
% ^^A ]]] End of subsection `***'.
%
% ^^A ]]] End of section `***'.
%
% \section{文章関係}^^A [[[
%
% \subsubsection{\texcommand{1}, \texcommand{2}, ...}^^A [[[
%
% \paragraph{Creating a List}^^A [[[
% タイトルは\vimhelp{list}より.
% \begin{vimcode}
let s:texStatementList=[
    \ ['\\$', '$'],
    \ ['\\&', '&'],
    \ ['\\#', '#'],
    \ ['\\_', '_'],
    \ ['\\%', '%'],
    \ ['\\1', '・'],
    \ ['\\2', 'ー'],
    \ ['\\3', '*'],
    \ ['\\4', '@']]
% \end{vimcode}
% ^^A ]]] End of paragraph `Creating a List'.
%
% \paragraph{Defining Syntax}^^A [[[
% タイトルの由来は\verb|:h syn-define|より.
%
% \href{run:/usr/share/nvim/runtime/syntax/tex.vim}{\texttt{tex.vim}} \myline{477--517}を参考にした.
%
% \doublequotes{cmd}の由来は\href{https://writingexplained.org/english-abbreviations/command}{ここ}や
% \href{https://en.wikipedia.org/wiki/Cmd.exe}{\texttt{cmd.exe}}より.
% \begin{vimcode}
for texcmd in s:texStatementList
    exe "syn match texStatement '".texcmd[0]."' conceal cchar=".texcmd[1]
endfor
% \end{vimcode}
% ^^A ]]] End of paragraph `Defining Syntax'.
%
% ^^A ]]] End of subsubsection `\1, \2, ...'.
%
% \subsubsection{\texcommand{doublequotes}系}^^A [[[
% タイトルはテキトーにつけた.
%
% \paragraph{Creating a List}^^A [[[
% \begin{vimcode}
let s:texStatementList=[
    \ ['doublequotes'	, '❝'	, '❞'],
%    \ ['section'		, '❛'	, '❜'],
%    \ ['subsection'		, '『'	, '』'],
    \ ['texcommand'		, '\'	, ' '],
    \ ['title'			, '❝'	, '❞']]
% \end{vimcode}
% ^^A ]]] End of paragraph `Creating a List'.
%
% \paragraph{Defining Syntax}^^A [[[ OK (2020-01-02T05:22:00)
% \begin{vimcode}
for texcmd in s:texStatementList
    exe "syn match ".texcmd[0]."CommandRight contained '\\\\".texcmd[0]."{.\\{-}' conceal cchar=".texcmd[1]
    exe "syn match ".texcmd[0]."CommandLeft contained '}' conceal cchar=".texcmd[2]
    exe "syn match texStatement '\\\\".texcmd[0]."{.\\{-}}'"
        \ "contains=".texcmd[0]."CommandRight,".texcmd[0]."CommandLeft"
"       \ ."contains=".texcmd[0]."CommandRight,".texcmd[0]."CommandLeft"
endfor
% \end{vimcode}
% ^^A ]]] End of paragraph `Defining Syntax'.
%
% ^^A ]]] End of subsubsection `***'.
%
% \subsubsection{\texcommand{***ref}}^^A [[[
% \doublequotes{式(...)}のように日本語表示にした理由は,その方が文字数が絞れるから.
% 英語だと\doublequotes{Eq.(...)},\doublequotes{Def.(...)},\doublequotes{Prop.(...)}など
% 文字数が様々なので,コードが複雑になってしまう.
%
% \paragraph{Creating a List}^^A [[[ OK (2020-01-02T09:46:16)
% \begin{vimcode}
let s:texStatementList=[
    \ ['axiom'	, '公理'],
    \ ['def'	, '定義'],
    \ ['eq'		, '式'],
    \ ['lemma'	, '補題'],
    \ ['prf'	, '式'],
    \ ['prop'	, '命題'],
    \ ['thm'	, '定理']]
% \end{vimcode}
% ^^A ]]] End of paragraph `Creating a List'.
%
% \paragraph{Defining Syntax}^^A [[[ OK (2020-01-02T11:28:27)
% \verb|leftParenthesis, rightParenthethesis|は$C(X), L(X), U(X)$の箇所で定義したものを使い回す.
%
% 文字数を得るには\verb|strlen|関数ではなく\verb|strchars|関数を用いること(
% \href{https://renenyffenegger.ch/notes/development/vim/script/vimscript/functions/strlen}{vim script: strlen}
% または
% \href{https://renenyffenegger.ch/notes/development/vim/script/vimscript/functions/strchars}{vim script: strchars}
% より).
% \begin{vimcode}
% syn match defrefCommandDef contained "\\def" conceal cchar=定
% syn match defrefCommandRef contained "ref" conceal cchar=義
% syn match texStatement "\\defref{.\{-}}" contains=defrefCommandDef,defrefCommandRef,leftParenthesis,rightParenthethesis
%
for texcmd in s:texStatementList
    if strchars(texcmd[1]) == 1
        exe "syn match ".texcmd[0]."refCommand contained '\\\\".texcmd[0]."ref' conceal cchar=".texcmd[1]
        exe "syn match texStatement '\\\\".texcmd[0]."ref{.\\{-}}'"
            \ "contains=".texcmd[0]."refCommand,leftParenthesis,rightParenthethesis"
"           \ ."contains=".texcmd[0]."refCommand,leftParenthesis,rightParenthethesis"
    else
        let chars = split(texcmd[1], '\zs')
        exe "syn match ".texcmd[0]."refCommand".texcmd[0]." contained '\\\\".texcmd[0]."'"
        	\ "conceal cchar=".chars[0]
        exe "syn match ".texcmd[0]."refCommandRef contained 'ref' conceal cchar=".chars[1]
        exe "syn match texStatement '\\\\".texcmd[0]."ref{.\\{-}}'"
        	\ "contains=".texcmd[0]."refCommand".texcmd[0].","
        	\ .texcmd[0]."refCommandRef,leftParenthesis,rightParenthethesis"
    endif
endfor
% \end{vimcode}
% ^^A ]]] End of paragraph `Defining Syntax'.
%
% ^^A ]]] End of subsubsection `\***ref'.
%
% \subsubsection{\texcommand{href}系}^^A [[[
% タイトルはテキトーにつけた.
%
% \paragraph{Creating a List}^^A [[[
% \begin{vimcode}
let s:texStatementList=[
    \ ['href'   , '👉'  , '👈'],
    \ ['urlref' , '👉'  , '👈'],]
% \end{vimcode}
% ^^A ]]] End of paragraph `Creating a List'.
%
% \paragraph{Defining Syntax}^^A [[[
% \begin{vimcode}
% syn region texRefZone matchgroup=texStatement start="\\href{.\{-}}{"
	% \ end="}\|%stopzone\>" contains=@texRefGroup cchar=寝
%
% syn region texRefZone matchgroup=texStatement start="\\href{.\{-}}{" end="}\|%stopzone\>" contains=@texRefGroup cchar=寝
%
% syn region texRefZone matchgroup=texStatement start="\\href{.\{-}}{" end="}\|%stopzone\>" cchar=寝
% syn region texRefZone matchgroup=texStatement start="\\href{" end="}\|%stopzone\>" cchar=寝
%





% 以下で上手く行った
" syn match hrefCommandLeft contained '\\href{.\{-}}{.\{-}' conceal cchar=👉
" syn match hrefCommandRight contained '}' conceal cchar=👈
" syn match texStatement '\\href{.\{-}}{.\{-}}' contains=hrefCommandLeft,hrefCommandRight




for texcmd in s:texStatementList
  exe "syn match ".texcmd[0]."CommandLeft contained '\\\\".texcmd[0]."{.\\{-}}{.\\{-}' conceal cchar=".texcmd[1]
  exe "syn match ".texcmd[0]."CommandRight contained '}' conceal cchar=".texcmd[2]
  exe "syn match texStatement '\\\\".texcmd[0]."{.\\{-}}{.\\{-}}' contains=".texcmd[0]."CommandLeft,".texcmd[0]."CommandRight"
endfor





% \end{vimcode}
% ^^A ]]] End of paragraph `Defining Syntax'.
%
% ^^A ]]] End of subsubsection `***'.
%
% \subsubsection{\texcommand{verb}系}^^A [[[
% \verb|minted|の\verb|\mint|で定義されたコマンドもこれに該当する.
%
% \paragraph{Creating a List}^^A [[[ OK (2020-01-04T02:33:05)
% \doublequotes{delimiter}の由来およびその略\doublequotes{delim}は\texdoc{minted}の\verb|\mint|より.
%
% デリミタ(delimiter)とは,複数の要素を列挙する際に,
% 要素の区切りとなる記号や特殊文字(の並び)のこと\mycite[keyword=delimiter]{e-words}.
% \begin{vimcode}
let s:texCommandList=[
    \ 'bash',
    \ 'latex',
    \ 'myverb',
    \ 'verb']
%
let s:texDelimiterList=[
    \ '|',
    \ '/',
    \ '?']
% \end{vimcode}
% ^^A ]]] End of paragraph `Creating a List'.
%
% \paragraph{Defining Syntax}^^A [[[
%
% \begin{vimcode}
% for texcmd in s:texCommandList
	% for texdelim in s:texDelimiterList
		% exe "syn match ".texcmd."CommandLeft contained '\\\\".texcmd.texdelim."' conceal"
		% exe "hi ".texcmd."CommandLeft ctermfg=118 ctermbg=0 cterm=bold,underline"
		% exe "syn match ".texcmd."CommandRight contained '".texdelim."' conceal"
		% exe "hi ".texcmd."CommandRight ctermfg=118 ctermbg=0 cterm=bold,underline"
		% exe "syn match ".texcmd."Command '\\\\".texcmd.texdelim.".\{-}".texdelim."'"
			% \ "contains=".texcmd."CommandLeft,".texcmd."CommandRight"
		% exe "hi ".texcmd."Command ctermfg=0 ctermbg=118 cterm=bold,underline"
	% endfor
% endfor



syn match verbCommandLeft contained "\\verb|" conceal
hi verbCommandLeft ctermfg=118 cterm=bold
syn match verbCommandRight contained "|" conceal
hi verbCommandRight ctermfg=118 cterm=bold
syn match verbCommand "\\verb|.\{-}|" contains=verbCommandLeft,verbCommandRight
hi verbCommand ctermfg=118 cterm=bold,underline




% 下記でうまくいくが,\section以降ではうまくいかない
% syn match bashinlineCommandLeft contained "\\bashinline{" conceal cchar=$
% hi bashinlineCommandLeft ctermfg=118 cterm=bold
% syn match bashinlineCommandRight contained "}" conceal
% hi bashinlineCommandRight ctermfg=118 cterm=bold
% syn match bashinlineCommand "\\bashinline{.\{-}}" contains=bashinlineCommandLeft,bashinlineCommandRight
% hi bashinlineCommand ctermfg=118 cterm=bold,underline

% syn region texZone start="\\bashinline{" end="}\|%stopzone\>"
syn region texZone start="\\bashinline\*{" end="}\|%stopzone\>"

% syn region texZone start="\\pythoninline{" end="}\|%stopzone\>"

syn region texZone start="\\.*inline{" end="}\|%stopzone\>"

% \end{vimcode}
% ^^A ]]] End of paragraph `Defining Syntax'.
%
% ^^A ]]] End of subsubsection `\verb'.
%
% \subsubsection{\texcommand{myarrow}}^^A [[[
% \begin{vimcode}
syn match texStatement "\\myarrow" conceal cchar=➞
% \end{vimcode}
% ^^A ]]] End of subsubsection `\myarrow'.
%
% \subsubsection{\texcommand{mybecause}}^^A [[[
% 
% \href{https://stackoverflow.com/questions/8309815/vim-conceal-with-more-than-one-character}{ここ}を参考にした.
%
% \verb|:syn-concealends|を使えばできそう.
%
% \verb|\mybecause{\eqref{***}}|とすると,意図したように表示されない.
%
% \begin{vimcode}
% syn match mybecauseCommand contained "\\mybecause" conceal cchar=(
% syn match mybecauseLeft contained "{" conceal cchar=∵
% syn match mybecauseRight contained "}" conceal cchar=)
% syn match texMathSymbol "\\mybecause\s*{.\{-}}" contains=mybecauseCommand,mybecauseLeft,mybecauseRight
% syn match texMathSymbol "\\mybecause\s*{.\{-}}" contains=mybecauseCommand,mybecauseLeft,.\{-},mybecauseRight
%
% syn region texMathSymbol start="\\mybecause{" end="}" contains=mybecauseCommand,mybecauseLeft,mybecauseRight
%



syn match texMathSymbol "\\mybecause" conceal cchar=∵



% \end{vimcode}
% ^^A ]]] End of subsubsection `****'.
%
%
%
%
% \subsubsection{\texcommand{subfile}}^^A [[[
% \href{run:/usr/share/nvim/runtime/syntax/tex.vim}{\texttt{tex.vim}} \myline{283}の
% \verb/\include(graphics|list)/を参考にした.
%
% \begin{vimcode}
syn match texInputFile "\\subfile\=\(\[.\{-}\]\)\=\s*{.\{-}}" contains=texStatement,texInputCurlies,texInputFileOpt
% \end{vimcode}
% ^^A ]]] End of subsubsection `subfile'.
%
% ^^A ]]] End of section `***'.
%
% \section{Math}^^A [[[
%
% \subsubsection{Including New Math Group}^^A [[[
% タイトルは\verb|:h tex-math|を参考にした.
%
% コードは\verb|:h tex-math|と
% \href{run:/usr/share/nvim/runtime/syntax/tex.vim}{\texttt{tex.vim}} \myline{439}を参考にした.
%
% "B"にした理由は特にない.上手くいったので暫定OK.あとで調べよ.
%
% 3つ目の引数は\verb|*|付きも定義するかどうかを指定するフラグ(\vimhelp{tex-math}).
%
% \begin{vimcode}
call TexNewMathZone("B","align",1)
call TexNewMathZone("B","gather",1)
call TexNewMathZone("B","mycalculation",1)
% \end{vimcode}
% ^^A ]]] End of subsubsection `Including New Math Group'.
%
% \subsubsection{数}^^A [[[
% タイトルはテキトーにつけた.
%
% \paragraph{Creating a List}^^A [[[ OK (2020-01-02T09:49:41)
% タイトルは\verb|:h list|より.
% \begin{vimcode}
let s:texMathList=[
    \ ['N', 'N'],
    \ ['Z', 'Z'],
    \ ['Q', 'Q'],
    \ ['R', 'R'],
    \ ['C', 'C']]
% \end{vimcode}
% ^^A ]]] End of paragraph `Creating a List'.
%
% \paragraph{Defining Syntax}^^A [[[
% \href{run:/usr/share/nvim/runtime/syntax/tex.vim}{\texttt{tex.vim}} \myline{677--856}より.
%
% \verb|syn match texMathSymbol "\\R" conceal cchar=R|とすると\verb|\Rightarrow|も変換されてしまう.
% \verb|"\\R\>"|とすることで防げる.
%
% \begin{vimcode}
for texmath in s:texMathList
    exe "syn match texMathSymbol '\\\\".texmath[0]."\\>' contained conceal cchar=".texmath[1]
endfor
% \end{vimcode}
% ^^A ]]] End of paragraph `Defining Syntax'.
%
% ^^A ]]] End of subsubsection `****'.
%
% \subsubsection{関数}^^A [[[
% タイトルの由来は\mycite[page=409]{teach-yourself-latex2e}より.
%
% \paragraph{Creating a List}^^A [[[
% 文字数は任意.
%    \begin{macrocode}
let s:texMathList=[
    \ 'cos',
    \ 'inf',
    \ 'lim',
    \ 'max',
    \ 'min',
    \ 'sin',
    \ 'sup',
    \ 'tan',
    \ 'tanh']
%    \end{macrocode}
% ^^A ]]] End of paragraph `Creating a List'.
%
% \paragraph{Defining Syntax}^^A [[[ OK (2020-01-02T02:25:27)
% タイトルの由来は\verb|:h syn-define|より.
%
% \href{https://groups.google.com/forum/#!topic/vim_dev/0RurxXUlhbA}{Getting a character from a string.}
%
% \href{https://renenyffenegger.ch/notes/development/vim/script/vimscript/functions/range}{vim script: range}
%
% \verb|+=|は使えない.また,\verb|+|は数値にしか使えない.文字列の連結は\verb|.|を使うこと.
% \begin{vimcode}
% syn match maxCommandM contained "\\m" conceal cchar=m
% syn match maxCommandA contained "a" conceal cchar=a
% syn match maxCommandX contained "x" conceal cchar=x
% syn match texMathSymbol "\\max\>" contains=maxCommandM,maxCommandA,maxCommandX
%
for texmath in s:texMathList
    let chars = split(texmath, '\zs')
    for i in range(strchars(texmath))
        if i == 0
            exe "syn match ".texmath."Command".chars[i]." contained '\\\\".chars[i]."'"
                \ "conceal cchar=".chars[i]
            let temp = texmath."Command".chars[i]
        else
            exe "syn match ".texmath."Command".chars[i]." contained '".chars[i]."'"
                \ "conceal cchar=".chars[i]
            let temp = temp.",".texmath."Command".chars[i]
        endif
    endfor
    exe "syn match texMathSymbol '\\\\".texmath."\\>' contains=".temp
endfor
% \end{vimcode}
% ^^A ]]] End of paragraph `Defining Syntax'.
%
% ^^A ]]] End of subsubsection `関数'.
%
% \subsubsection{$C(X), L(X), U(X)$}^^A [[[
% タイトルはテキトーにつけた.
%
% \paragraph{Creating a List}^^A [[[ OK (2020-01-02T09:57:01)
%    \begin{macrocode}
let s:texMathList=[
    \ ['cut'        , 'C'],
    \ ['lowerbound' , 'L'],
    \ ['upperbound' , 'U']]
%    \end{macrocode}
% ^^A ]]] End of paragraph `Creating a List'.
%
% \paragraph{Defining Syntax}^^A [[[ OK (2020-01-02T03:18:37)
% タイトルの由来は\verb|:h syn-define|より.
%
% \href{https://stackoverflow.com/questions/8309815/vim-conceal-with-more-than-one-character}{ここ}を参考にした.
%
% \verb|:syn-concealends|を使えばできそう.
%
% \doublequotes{left[right] parenthesis}の由来は
% \href{https://ja.wikipedia.org/wiki/%E6%8B%AC%E5%BC%A7#%E4%B8%B8%E6%8B%AC%E5%BC%A7%EF%BC%88%EF%BC%89}{ここ}より.
%
%    \begin{macrocode}
syn match leftParenthesis contained "{" conceal cchar=(
syn match rightParenthethesis contained "}" conceal cchar=)
%
% syn match cutCommand contained "\\cut" conceal cchar=C
% syn match texMathSymbol "\\cut{.\{-}}" contains=cutCommand,leftParenthesis,rightParenthethesis
%
for texmath in s:texMathList
    exe "syn match ".texmath[0]."Command contained '\\\\".texmath[0]."'"
        \ "conceal cchar=".texmath[1]
    exe "syn match texMathSymbol '\\\\".texmath[0]."{.\\{-}}' contains="
        \ .texmath[0]."Command,leftParenthesis,rightParenthethesis"
endfor
%    \end{macrocode}
% ^^A ]]] End of paragraph `Defining Syntax'.
%
% ^^A ]]] End of subsubsection `****'.
%
% \subsubsection{表示の修正}^^A [[[
%
% \paragraph{Creating a List}^^A [[[
% タイトルは\verb|:h list|より.
%    \begin{macrocode}
let s:texMathList=[
    \ ['emptyset'		, '∅'],
    \ ['exists'			, '∃'],
    \ ['forall'			, '∀'],
    \ ['in'				, '∈'],
    \ ['land'			, '∧'],
    \ ['Leftrightarrow'	, '⇔'],
    \ ['lor'			, '∨'],
    \ ['neq'			, '≠'],
    \ ['notin'			, '∉'],
    \ ['Rightarrow'		, '⇒'],
    \ ['subset'			, '⊂'],
    \ ['subseteq'		, '⊆']]
%    \end{macrocode}
% ^^A ]]] End of paragraph `Creating a List'.
%
% \paragraph{Defining Syntax}^^A [[[ OK (2020-01-02T23:41:05)
% \href{https://stackoverflow.com/questions/8309815/vim-conceal-with-more-than-one-character}{ここ}を参考にした.
%
% Listのindexについては\href{https://learnvimscriptthehardway.stevelosh.com/chapters/35.html#indexing}{ここ}を参考にした.
%
% join関数は\href{https://renenyffenegger.ch/notes/development/vim/script/vimscript/functions/join}{ここ}を参考にした.
%
% \verb|:syn-concealends|を使えばできそう.
% \begin{vimcode}
% syn match inCommand contained "\\i" conceal cchar=∈
% syn match inSpace contained "n" conceal cchar=.
% syn match texMathSymbol "\\in\>" contains=inCommand,inSpace
let tempSpace = ' '
for texmath in s:texMathList
    let chars = split(texmath[0], '\zs')
    let tempNum = strchars(texmath[0]) - 2
    let tempTexCommand = join(chars[0:tempNum], '')
    exe "syn match ".texmath[0]."Command contained '\\\\".tempTexCommand."'"
        \ "conceal cchar=".texmath[1]
    exe "syn match ".texmath[0]."CommandSpace contained '".chars[-1]."'"
        \ "conceal cchar=".tempSpace
    exe "syn match texMathSymbol '\\\\".texmath[0]."\\>'"
        \ "contains=".texmath[0]."Command,".texmath[0]."CommandSpace"
endfor
% \end{vimcode}
% ^^A ]]] End of paragraph `Defining Syntax'.
%
% ^^A ]]] End of subsubsection `****'.
%
% ^^A ]]] End of section `Math'.
%
%
%
%
% \section{***code environment}^^A [[[
% 以下は\verb|/usr/share/nvim/runtime/syntax/tex.vim|の\verb|verbatim|の箇所を参考にした.
%
% \begin{vimcode}
syn region texZone start="\\begin{.*code}" end="\\end{.*code}\|%stopzone\>"
% syn region texProgCode start="\\begin{.*code}" end="\\end{.*code}\|%stopzone\>"

% hi def link texProgCode PreCondit
% \end{vimcode}
% ^^A ]]] End of section `***code environment'.
%
%
%
%
%
%
%
% \begin{vimcode}

% hi link texMathSymbol Statement
hi texMathSymbol ctermfg=135 cterm=underline
hi texStatement ctermfg=135 cterm=underline
% hi doublequotesCommandRight ctermfg=9 ctermbg=20 cterm=bold
hi doublequotesCommandRight cterm=bold



hi dtxTest ctermfg=0 ctermbg=208 cterm=underline
syn match dtxTest '%<.*>'



% \end{vimcode}
%
%
%
%
%
%
%
%
%
%
%
%
%
%
%
%
% 以下は\href{run:/usr/share/nvim/runtime/syntax/tex.vim}{\texttt{tex.vim}} \myline{384}を参考にした.
% \begin{vimcode}
syn region texMathDelim	matchgroup=texTypeStyle start="\\mymathbold\s*{" matchgroup=texTypeStyle  end="}" concealends contains=@texBoldGroup,@Spell
% \end{vimcode}
%
%
%
%
% \iffalse
%</tex.vim>
% \fi
% ^^A ]]] End of part `tex.vim'.
%
% ^^A End of file `syntax.dtx'.
