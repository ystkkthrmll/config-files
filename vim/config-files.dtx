%
% ^^A Configuring DocStrip [[[
% \iffalse
%<*driver>
\input mydocstrip
\mygenerate[output-files/config-files]{
  init.vim,
  .NERDTreeBookmarks.home,
  .NERDTreeBookmarks.office,
}
\myendbatchfile
\mydriver[
  show-notes,
]
%</driver>
% \fi
% ^^A ]]] End of Configuring DocStrip
%
% \mytitle{Neovim configuration file}
%
% \begin{concealableitemize}^^A [[[
% \1 The origin of `Nvim config file'
%   \2 \verb|The Nvim config file is named "init.vim"| \vimhelp{config}
% \end{concealableitemize}^^A ]]]
%
% \begin{abstract}^^A [[[
% \begin{myitemize}
% \1 \verb|init.vim|
%   \2 Neovim configuration file
% \1 \verb|.NERDTreeBookmarks|
% \1 DGX上でneovimが使えるか確認したところ,snapのnvimがあるのでインストールを試みた.
% しかし,\bashinline{snap install nvim}を実行するとsudoを要求されるため断念
% したがって,\verb|.vimrc|を作成する
%   \2 luaがなく,neocomplete.vimが使えないっぽい
% \end{myitemize}
% \end{abstract}^^A ]]]
%
% \begin{mynote}^^A [[[
% \begin{myitemize}
% \1 QuickFixについて調べよ(エラー箇所に直ぐに行ける?)
% \1 pythonのためのvimプラグインを調査せよ.
% \1 \href{https://tabnine.com/}{tabnine}について調べよ.
% \1 \href{https://github.com/Shougo/deoppet.nvim}{deoppet.nvim}というものがあるがまだ開発中らしい.
% \1 unite.vim, denite.nvim, vim-latex-live-preview, について調べよ
% \1 vimで\verb/% [[[/や\verb/% ]]]/などを挿入するコマンドについて調べよ
% \1 
% \1 
% \end{myitemize}
% \end{mynote}^^A ]]]
%
% \begin{mytip}^^A [[[
% \begin{myitemize}
% \0 便利機能
% \1 \verb|:e(dit) $MYVIMRC|で\verb|init.vim|をすぐに編集できる\vimhelp{vimrc-intro}
% \1 \verb|:terminal|でneovim内でターミナルを起動できる
% \1 \verb|:!cmd|で外部プログラムを実行できる
% \urlref{https://nanasi.jp/articles/howto/editing/external-program.html}{参照先}
%   \2 例:\verb|:!texdoc xcolor|
% \1 \verb|:foldopen|で折畳を開ける
% \1 vimからクリップボードにコピーするにはvisualで領域選択しCtrl+C
% \1 If you are using Vim 7.4 or later, in normal mode, put your cursor below the URL,
% then click \verb|gx|, the URL will be opened in browser automatic
% \urlref{https://stackoverflow.com/questions/9458294/open-url-under-cursor-in-vim-with-browser}{Open URL under cursor in Vim with browser}.
% \0 留意事項
% \1 \verb|.vim|はDropbox上に置かない方が良い気がする.なぜなら,サイズが大きいから.ローカルに置け.
% \1 vimでファイル編集するのは,普通vimを開いてから,unite.vimなどで移動編集する.
% \1 vimtexのTag navigationは使えそう.(\verb|:help vimtex|のTag navigationを参照)
%   \2 One may navigate by tags with the \verb|CTRL-]| mapping, e.g from
%   \verb|\eqref{eq:example}| to the corresponding \verb|\label{eq:example}|.
% \1 
% \end{myitemize}
% \end{mytip}^^A ]]]
%
% \mytableofcontents
%
% \part{Notes}^^A [[[
% \begin{myitemize}
% \1 SpaceVimは以下理由により使用しない
%   \2 ほとんどLayerで管理されているからカスタマイズしにくい
%   \2 \LaTeX でコンパイルするときcleanしてくれない.
%   なのでquickrunで処理したいが,今度はsubfilesパッケージを使うとquickrunがエラーで機能しない
%   \2 conceal機能が上手くいかない.insertモードからnormalモードに戻るとconceal機能が働かない
% \end{myitemize}
% ^^A ]]] End of part `Notes'.
%
% \part{\myverb{init.vim}}^^A [[[
% \begin{myitemize}
% \1 The origin of \verb|~/.config/nvim/init.vim|
%   \2 The Nvim config file is named "init.vim", located at:
%   Unix \verb|~/.config/nvim/init.vim| \vimhelp{config}
%   \2 This file is always used and is recommended:
%   \verb|~/.config/nvim/init.vim| (Unix and OSX) \vimhelp{vimrc-intro}
% \end{myitemize}
%
% \iffalse
%<*init.vim>
% \fi
%
% \section{set nocompatible}^^A [[[
% \begin{myitemize}
% \1 コードは\vimhelp{dein-examples}より
% \1 \verb|set nocompatible|とは\verb|'compatible'|オプションを無効にするため設定です。
% では\verb|'compatible'|オプションとは何なのかというと『VimをなるべくVi互換にする』ためのオプションになります。
% つまり\verb|'compatible'|オプションが有効な場合は、『VimがVi互換となっている』為、
% 『Vimの便利な機能』が使えません。
% なのでVimをVimらしく使うためには\verb|'compatible'|オプションを無効にするために
% \verb|set nocompatible|をする必要があります
% \urlref{http://secret-garden.hatenablog.com/entry/2017/12/02/000156}{【一人 vimrc Advent Calendar 2017】set nocompatibleとは【2日目】}。
% \end{myitemize}
%
% \begin{vimcode}
% if &compatible
%   set nocompatible
% endif
% \end{vimcode}
% ^^A ]]] End of section `set nocompatible'.
%
% \section{plugin}^^A [[[
%
% \begin{concealableitemize}^^A [[[
% \1 The origin of `plugin'
%   \2 \vimhelp{plugin}
%   \2 
% \end{concealableitemize}^^A ]]]
%
% \begin{vimcode}
call plug#begin('~/.vim/plugged')
% \end{vimcode}
%
% \subsection{File explorer}^^A [[[
% \begin{myitemize}
% \1 \viminline{let g:NERDTreeQuitOnOpen = 1}
%   \2 This setting governs whether the NERDTree window or the bookmarks table
%   closes after opening a file with the \verb|NERDTree-o|, \verb|NERDTree-i|,
%   \verb|NERDTree-t| and \verb|NERDTree-T| mappings \vimhelp{NERDTreeQuitOnOpen}.
% \1 \viminline{nnoremap <silent>f :NERDTreeToggle<CR>}
%   \2  \vimhelp{NERDTreeToggle}
%   \2 本来のfは?
% \1 
% \end{myitemize}
%
% \begin{vimcode}
Plug 'preservim/nerdtree', {'on': 'NERDTreeToggle'}
let g:NERDTreeQuitOnOpen = 1
nnoremap <silent>f :NERDTreeToggle<CR>
% \end{vimcode}
% ^^A ]]] End of subsection `File explorer'.
%
% \subsection{mhinz/vim-signify}^^A [[[
% \begin{vimcode}
Plug 'mhinz/vim-signify'
% \end{vimcode}
% ^^A ]]] End of subsection `***'.
%
% \subsection{preservim/nerdcommenter}^^A [[[
% \begin{myitemize}
% \1 \href{https://github.com/preservim/nerdcommenter#nerd-commenter-}{Comment functions so powerful--no comment necessary.}
% \1 \viminline{let g:NERDSpaceDelims = 1}
%   \2 If you want spaces to be added then set \viminline{NERDSpaceDelims} to 1 in your vimrc
%   \vimhelp{NERDSpaceDelims}.
% \1 \viminline{map <silent>c <Plug>NERDCommenterToggle}
%   \2 Toggles commenting of the lines selected \vimhelp{NERDCommenterToggle}
%   \2 \verb|nnoremap|だと上手くいかない
%   \2 本来のcはdeleteだと思う.調べよ
% \end{myitemize}
%
% \begin{vimcode}
Plug 'preservim/nerdcommenter'
let g:NERDSpaceDelims = 1
% nnoremap <silent>c <Plug>NERDCommenterToggle
% nnoremap <silent>c <Leader>c<Space>
map <silent>c <Plug>NERDCommenterToggle
% \end{vimcode}
% ^^A ]]] End of subsection `preservim/nerdcommenter'.
%
% \subsection{Shougo/deoplete.nvim}^^A [[[
% \begin{myitemize}
% \1 Dark powered asynchronous completion framework for Neovim/Vim8 \vimhelp{deoplete.txt}.
% \1 \viminline{let g:deoplete#enable_at_startup = 1}
%   \2 Deoplete gets started automatically when Neovim/Vim starts if this value is 1
%   \vimhelp{g:deoplete#enable_at_startup}.
% \1 \viminline{call deoplete#custom#option('min_pattern_length', 1)}
%   \2 The default number of the input completion at the time of key input automatically
%   \vimhelp{min_pattern_length}.
% \1 
% \end{myitemize}
%
% \begin{vimcode}
Plug 'Shougo/deoplete.nvim'
let g:deoplete#enable_at_startup = 1
" call deoplete#custom#option('min_pattern_length', 1)
inoremap <expr> <CR> (pumvisible() ? deoplete#close_popup() : "\<CR>")
inoremap <expr><TAB> pumvisible() ? "\<C-n>" : "\<TAB>"
inoremap <expr><C-h> deoplete#smart_close_popup()."\<C-h>"
inoremap <expr><BS> deoplete#smart_close_popup()."\<C-h>"
% \end{vimcode}
% ^^A ]]] End of subsection `Shougo/deoplete.nvim'.
%
% \subsection{Shougo/vimproc.vim}^^A [[[
% \begin{myitemize}
% \1 非同期処理.quickrunには必須のプラグイン.
% \1 初めて使うときは,\verb|.../Shougo/vimproc.vim|上で\verb|make|をする必要あり
% \vimhelp{vimproc-install}
% \1 \verb|call dein#add('Shougo/vimproc.vim', {'build': 'make'})|は\verb|vimproc.txt|より.
% \1 \verb|build='make'|はテキトーにつけた(どっかネットの情報よりつけた気がする.必要かはわからない)
% \end{myitemize}
%
% \begin{vimcode}
Plug 'Shougo/vimproc.vim'
% \end{vimcode}
% ^^A ]]] End of subsection `Shougo/vimproc.vim'.
%
% \subsection{thinca/vim-quickrun}^^A [[[
% \begin{myitemize}
% \1 Run a command and show its result quickly.
% \1 \href{http://qiita.com/ssh0/items/4aea2d3849667717b36d}{ここ}や
% \href{http://auewe.hatenablog.com/entry/2013/12/25/033416}{ここ}を参考にした
% \1 \verb|/hook/sweep/files|は削除する一時ファイルのリスト(cf. \verb/quickrun.jax/).
% \1 \verb/.tmp/はshowexplのLTXexample環境で利用される.
% \1 \verb/\r/で実行
% \1 \verb|'runner': 'vimproc'|をつけることにより,非同期処理できる(cf. \url{http://qiita.com/uplus_e10/items/2a75fbe3d80063eb9c18}).
% \1 \verb|'runner': 'vimproc'|をつけると,なぜか\verb|hook/sweep/files|が機能しない.
% \1 \verb|'cmdopt': '-pvc'|にすると,ファイルが保存されているたびに自動コンパイルしてくれる.
% しかし,編集できなくなることがあるのでやめる.
% \1 基本構造
%   \2 \vimhelp{quickrun-examples}
% \1 
% \1 
% \end{myitemize}
%
% \begin{vimcode}
Plug 'thinca/vim-quickrun'
nnoremap \r :QuickRun<CR>
% \end{vimcode}
%
% \subsubsection{部分コンパイル}^^A [[[
% 一応,載せておく.
%    \begin{macrocode}
% 	let g:quickrun_config.tmptex = {
% 		\ 'exec': [
% 		\			'mv %s %a/tmptex.latex',
% 		\			'latexmk -pv -output-directory=%a %a/tmptex.latex',
% 		\			],
% 		\ 'args': expand("%:p:h:gs?\\\\?/?"),
% 		\ 'outputter': 'error',
% 		"\ 'outputter/error/error': 'buffer',
% 		\ 'outputter/error/error': 'quickfix',
% 		\ 'hook/eval/enable': 1,
% 		\ 'hook/eval/cd': "%s:r",
% 		\
% 		\ 'hook/eval/template':	 '\documentclass{ltjsarticle}'
% 		\						.'\usepackage[colorlinks=true,pdfencoding=auto]{hyperref}'
% 		\						.'\usepackage{mymath}'
% 		\						.'\begin{document}'
% 		\						.'%s'
% 		\						.'\end{document}',
% 		\
% 		\ 'hook/sweep/files': [
% 		\						'%a/tmptex.latex',
% 		\						'%a/tmptex.aux',
% 		\						'%a/tmptex.bbl',
% 		\						'%a/tmptex.blg',
% 		\						'%a/tmptex.fdb_latexmk',
% 		\						'%a/tmptex.fls',
% 		\						'%a/tmptex.idx',
% 		\						'%a/tmptex.ilg',
% 		\						'%a/tmptex.ind',
% 		\						'%a/tmptex.log',
% 		\						'%a/tmptex.out',
% 		\						'%a/tmptex.toc',
% 		\						'%a/tmptex.tmp',
% 		\						],
% 		\}
%    \end{macrocode}
% ^^A ]]] End of subsubsection `部分コンパイル'.
%
% ^^A ]]] End of subsection `vim-quickrun'.
%
% \subsection{Townk/vim-autoclose}^^A [[[
% \begin{vimcode}
Plug 'Townk/vim-autoclose'
% \end{vimcode}
% ^^A ]]] End of subsection `Townk/vim-autoclose'.
%
% \subsection{Snippets}^^A [[[
% g:UltiSnipsSnippetDirectories
% \begin{vimcode}
Plug 'SirVer/ultisnips'

" Snippets are separated from the engine. Add this if you want them:
Plug 'honza/vim-snippets'

" Trigger configuration. You need to change this to something other than <tab> if you use one of the following:
" - https://github.com/Valloric/YouCompleteMe
" - https://github.com/nvim-lua/completion-nvim
let g:UltiSnipsExpandTrigger="<c-k>"
let g:UltiSnipsJumpForwardTrigger="<c-k>"
let g:UltiSnipsJumpBackwardTrigger="<c-z>"

" If you want :UltiSnipsEdit to split your window.
let g:UltiSnipsEditSplit="vertical"

let g:UltiSnipsSnippetDirectories = [$HOME.'/Dropbox/config-files/vim/output-files/snippets']

% \end{vimcode}
% ^^A ]]] End of subsection `Snippets'.
%
% \subsection{neoclide/coc.nvim}^^A [[[
% \begin{myitemize}
% \1 \href{https://github.com/neoclide/coc.nvim#example-vim-configuration}{Example Vim configuration}
% \end{myitemize}
%
% \begin{vimcode}
" Plug 'neoclide/coc.nvim', {'branch': 'release'}
" Use tab for trigger completion with characters ahead and navigate
" NOTE: There's always complete item selected by default, you may want to enable
" no select by `"suggest.noselect": true` in your configuration file
" NOTE: Use command ':verbose imap <tab>' to make sure tab is not mapped by
" other plugin before putting this into your config
" inoremap <silent><expr> <TAB>
"      \ coc#pum#visible() ? coc#pum#next(1) :
"      \ CheckBackspace() ? "\<Tab>" :
"      \ coc#refresh()
"" inoremap <expr><S-TAB> coc#pum#visible() ? coc#pum#prev(1) : "\<C-h>"

" Make <CR> to accept selected completion item or notify coc.nvim to format
" <C-g>u breaks current undo, please make your own choice
" inoremap <silent><expr> <CR> coc#pum#visible() ? coc#pum#confirm()
"                              \: "\<C-g>u\<CR>\<c-r>=coc#on_enter()\<CR>"

" function! CheckBackspace() abort
"  let col = col('.') - 1
"  return !col || getline('.')[col - 1]  =~# '\s'
" endfunction

% \end{vimcode}
% ^^A ]]] End of subsection `neoclide/coc.nvim'.
%
%
%
%
%
% \begin{vimcode}

% Plug 'ycm-core/YouCompleteMe'

Plug 'tomasr/molokai'

% \end{vimcode}
%
%
%
%
%
%
% \begin{vimcode}
call plug#end()
% \end{vimcode}
%
% \subsection{molokai}^^A [[[
% \begin{myitemize}
% \1 注意:molokaiを最後にしないと表示がおかしくなる
% \1 色はmolokai.vimで用いられている色を使うこと
% \1 \viminline{hi Directory}
%   \2 NERDTreeで表示されるディレクトリ名のフォントを変更
%   \2 Ref: molokai.vim \myline{163, 154}
% \1 \viminline{hi MatchParen}
%   \2 Normal modeで対応する括弧の配色を逆にした
%   \2 Ref: molokai.vim \myline{180}
% \1 \viminline{hi NonText}
%   \2 テキスト外の色を変更
%   \2 Ref: molokai.vim \myline{231, 269}
% \1 \viminline{hi Visual}
%   \2 Visual modeで選択領域の色を変更
%   \2 Ref: molokai.vim \myline{223}
% \end{myitemize}
%
% \begin{vimcode}
syntax on
colorscheme molokai
hi Directory  ctermfg=118             cterm=italic
hi MatchParen ctermfg=208 ctermbg=233 cterm=bold
hi NonText    ctermfg=255 ctermbg=16
hi Visual     ctermfg=16  ctermbg=102 cterm=italic,bold
% \end{vimcode}
% ^^A ]]] End of subsection `molokai'.
%
% ^^A ]]] End of section `plugin'.
%
% \section{Settings}^^A [[[
% \doublequotes{Settings}の由来は.
%
% \subsection{Fold関係}^^A [[[
%
% \subsubsection{fold自動open}^^A [[[
% ファイルを開けたとき,カーソルのある行のfoldを自動で開ける
% \href{https://zenbro.github.io/2015/07/19/simulate-a-keypress-in-vim-insert-mode.html}{参照先}
% \begin{vimcode}
" if has("autocmd")
    " au BufReadPost * call feedkeys('zozozo')
    " au BufReadPost * foldopen
    " au BufReadPost * foldopen
" endif
% \end{vimcode}
% ^^A ]]] End of subsubsection `fold自動open'.
%
% ^^A ]]] End of subsection `Fold関係'.
%
% \subsubsection{scrolloff}^^A [[[
% Minimal number of screen lines to keep above and below the cursor.
%    \begin{macrocode}
set scrolloff=9999
%    \end{macrocode}
% ^^A ]]] End of subsubsection `scrolloff'.
%
% \subsection{Conceal関係}^^A [[[
%
% \subsubsection{concealcursor}^^A [[[
% ***************
% \begin{vimcode}
set concealcursor=""
% \end{vimcode}
% ^^A ]]] End of subsubsection `concealcursor'.
%
% \subsubsection{conceallevel}^^A [[[
% ***************
% \begin{vimcode}
set conceallevel=2
% \end{vimcode}
% ^^A ]]] End of subsubsection `conceallevel'.
%
% \subsubsection{listchars}^^A [[[
% \verb|:h listchars|
%
% \href{https://qiita.com/pollenjp/items/459a08a2cc59485fa08b}{参考}
% \begin{vimcode}
" set listchars=eol:↲,tab:>-,trail:-
set listchars=tab:>-,trail:*
set list
hi NonText ctermfg=red
% \end{vimcode}
% ^^A ]]] End of subsubsection `***'.
%
% ^^A ]]] End of subsection `Conceal関係'.
%
% \subsection{見た目関係}^^A [[[
% タイトル変更したい.
%
% \subsubsection{cursorline}^^A [[[ OK (2020-02-02T15:23:51)
% Highlight the screen line of the cursor with CursorLine \vimhelp{cursorline}.
%    \begin{macrocode}
set cursorline
%    \end{macrocode}
% ^^A ]]] End of subsubsection `cursorline'.
%
% \subsubsection{number}^^A [[[ OK (2020-02-02T15:27:14)
% Print the line number in front of each line \vimhelp{'number'}.
%    \begin{macrocode}
set number
%    \end{macrocode}
% ^^A ]]] End of subsubsection `number'.
%
% \subsubsection{ruler}^^A [[[ OK (2020-02-02T15:32:37)
% Show the line and column number of the cursor position, separated by a comma.
% When there is room, the relative position of the displayed text in the file is shown on the far right:
% \begin{center}
% \begin{tabular}{ll}
%   Top & first line is visible             \\
%   Bot & last line is visible              \\
%   All & first and last line are visible   \\
%   45\% & relative position in the file
% \end{tabular}
% \end{center}
%    \begin{macrocode}
set ruler
%    \end{macrocode}
% ^^A ]]] End of subsubsection `ruler'.
%
% ^^A ]]] End of subsection `見た目関係'.
%
% \subsubsection{nohlsearch}^^A [[[
%    \begin{macrocode}
set nohlsearch
%    \end{macrocode}
% ^^A ]]] End of subsubsection `foldmethod'.
%
% \subsubsection{lines, columns}^^A [[[
%    \begin{macrocode}
" set lines=150 columns=250
%    \end{macrocode}
% ^^A ]]] End of subsubsection `lines, columns'.
%
% \subsubsection{splitbelow, splitright}^^A [[[
% When on, splitting a window will put the new window below the current one \vimhelp{splitbelow}.
%
% When on, splitting a window will put the new window right of the current one \vimhelp{splitright}.
%    \begin{macrocode}
set splitbelow
set splitright
%    \end{macrocode}
% ^^A ]]] End of subsubsection `splitbelow, splitright'.
%
% \subsubsection{前回のカーソル位置からスタート*}^^A [[[
% 前回のカーソル位置からスタート
%    \begin{macrocode}
if has("autocmd")
    au BufReadPost * if line("'\"") > 1 && line("'\"") <= line("$") | exe "normal! g'\"" | endif
endif
%    \end{macrocode}
% ^^A ]]] End of subsubsection `***'.
%
% \subsubsection{モードによってカーソルの形を変える}^^A [[[
% Insert mode Normal mode cursor change (cf. neovim/wiki/FAQ)
% \begin{vimcode}
let $NVIM_TUI_ENABLE_CURSOR_SHAPE = 1
% \end{vimcode}
% ^^A ]]] End of subsubsection `モードによってカーソルの形を変える'.
%
% \subsubsection{IMEオフ}^^A [[[
% Insert mode から Normal mode に切り替わるとIMEを自動でオフにする.
% \begin{vimcode}
function! ImInActive()
    call system('fcitx-remote -c')
endfunction
inoremap <silent><Esc> <C-[>:call ImInActive()<CR>
% \end{vimcode}
% ^^A ]]] End of subsubsection `IMEオフ'.
%
% \subsubsection{Opening Vim help in a vertical split window}^^A [[[
% \href{https://stackoverflow.com/questions/630884/opening-vim-help-in-a-vertical-split-window}{Opening Vim help in a vertical split window}
%
% \begin{vimcode}
autocmd FileType help wincmd L
% \end{vimcode}
% ^^A ]]] End of subsubsection `Opening Vim help in a vertical split window'.
%
% \subsubsection{Others}^^A [[[
% \begin{myitemize}
% \1 \viminline{let g:tex_flavor = "latex"}
%   \2 常にlatexを想定(plain TeXとかではなく)
%   (cf. http://tex.stackexchange.com/questions/55397/vim-syntax-highlighting-of-partial-tex-file-used-with-include-is-incorrect)
%   \2 \verb|ftplugin/after/tex.vim|に書くと認識されない
% \end{myitemize}
%
% \begin{vimcode}
let g:tex_flavor = "latex"
% \end{vimcode}
% ^^A ]]] End of subsubsection `Others'.
%
% ^^A ]]] End of section `Settings'.
%
% \section{Key mapping}^^A [[[
% vimで開くファイルの種類に限らず,全ファイル共通して用いるkey mapping.
% \doublequotes{Key mapping}の由来は...
% 変更する可能性大あり.
%
% \verb|:map|でkey mapping一覧が表示される.
%
% \subsubsection{y}^^A [[[
% 本来は???
% copy
% \begin{vimcode}
nnoremap <silent>y yy
% \end{vimcode}
% ^^A ]]] End of subsubsection `***'.
%
% \subsubsection{q}^^A [[[
% 本来は???
% \begin{vimcode}
nnoremap <silent>q :q<CR>
% \end{vimcode}
% ^^A ]]] End of subsubsection `***'.
%
% \subsubsection{Space}^^A [[[
% 本来は???
% \begin{vimcode}
nnoremap <silent><Space> za
% \end{vimcode}
% ^^A ]]] End of subsubsection `Space'.
%
% \subsubsection{vsplit}^^A [[[
% \begin{myitemize}
% \1 The origin of \verb|Ctrl + \|
%   \2 \href{https://code.visualstudio.com/docs/getstarted/keybindings#_editorwindow-management}{workbench.action.splitEditor}
% \end{myitemize}
%
% \begin{vimcode}
nnoremap <silent><C-\> :vsplit<CR>
% \end{vimcode}
% ^^A ]]] End of subsubsection `vsplit'.
%
% \subsubsection{一時的}^^A [[[
%
% \begin{vimcode}
nnoremap <silent><CR> :w<CR>
% nnoremap <silent>+ 10<C-W>+ 20<C-W>>
% nnoremap <silent>- 10<C-W>- 20<C-W><
nnoremap <silent>+ 10<C-W>>
nnoremap <silent>- 10<C-W><
nnoremap <silent>= <C-W>=
nnoremap <silent>t gt
nnoremap <silent>m <C-W>w
nnoremap <silent>, :only<CR>
nnoremap <silent><C-@> :wqall<CR>

" test
inoremap <silent><C-J> <C-X><C-K>


nnoremap <silent><Leader>i :%s/outline/myitemize/gc<CR>


" a = add
" nnoremap <silent><buffer>\a :Gwrite<CR>


" https://stackoverflow.com/questions/1675688/make-vim-show-all-white-spaces-as-a-character
nnoremap <F5> :set list!<CR>



% \end{vimcode}
% ^^A ]]] End of subsubsection `***'.
%
% ^^A ]]] End of section `Key mapping'.
%
%
%
% \begin{myitemize}
% \1 open fold
%   \2 \href{https://maku77.github.io/vim/settings/autocmd.html}{autocmd で自動コマンドを登録する}
%   \2 \href{https://vi.stackexchange.com/questions/16045/recursively-open-all-folds-in-current-open-fold}{Recursively open all folds in current open fold?}
% \end{myitemize}
%
% \begin{vimcode}
% autocmd     BufNewFile,BufRead    *.*  :foldopen!
% autocmd VimEnter *.* :foldopen!
autocmd VimEnter *.* :normal 10za
% \end{vimcode}
%
%
% \iffalse
%</init.vim>
% \fi
%
% ^^A ]]] End of part `init.vim'.
%
% \part{\texttt{.NERDTreeBookmarks}}^^A [[[
%
% \section{Common}^^A [[[
% \iffalse
%<*.NERDTreeBookmarks.home|.NERDTreeBookmarks.office>
% \fi
%    \begin{macrocode}
config-files ~/Dropbox/config-files
$HOME ~
projects ~/projects
%    \end{macrocode}
% \iffalse
%</.NERDTreeBookmarks.home|.NERDTreeBookmarks.office>
% \fi
% ^^A ]]] End of section `Common'.
%
% \section{Home}^^A [[[
% \iffalse
%<*.NERDTreeBookmarks.home>
% \fi
%    \begin{macrocode}
computer ~/my-documents/computer
Documents ~/Documents
english ~/my-documents/english
misc ~/my-documents/misc
job ~/my-documents/job
math ~/my-documents/math/subfiles
physics ~/my-documents/physics
%    \end{macrocode}
% \iffalse
%</.NERDTreeBookmarks.home>
% \fi
% ^^A ]]] End of section `Home'.
%
% \section{Office}^^A [[[
% \iffalse
%<*.NERDTreeBookmarks.office>
% \fi
%    \begin{macrocode}
common ~/job/000-common-items
aidea ~/job/001-aidea---aisea
nagoya-cu ~/job/002-nagoya-city-university---head-ai
subaru ~/job/003-subaru---anomaly-detection
misc ~/job/999-miscellaneous
%    \end{macrocode}
% \iffalse
%</.NERDTreeBookmarks.office>
% \fi
% ^^A ]]] End of section `Office'.
%
% ^^A ]]] End of part `.NERDTreeBookmarks'.
%
% \appendix
%
% \section{Comparison of plugins for Neovim}^^A [[[
%
% \subsection{Plugin manager}^^A [[[
%
% \begin{concealableitemize}^^A [[[
% \1 The origin of `plugin manager'
%   \2 A minimalist Vim plugin manager
%   \urlref{https://github.com/junegunn/vim-plug}{junegunn/vim-plug}
%   \2 Dark powered Vim/Neovim plugin manager \vimhelp{dein.txt}
% \end{concealableitemize}^^A ]]]
%
% \begin{myitemize}
% \1 \href{https://qiita.com/nil2/items/ddcf23f1163d0abd805b}{Vimのプラグインマネージャの種類と選び方}
% \1 \href{https://www.slant.co/topics/1224/~best-plugin-managers-for-vim}{What are the best plugin managers for vim?}
% \1 
% \end{myitemize}
%
% \subsubsection{junegunn/vim-plug}^^A [[[
% \begin{myitemize}
% \1 \href{https://github.com/junegunn/vim-plug}{junegunn/vim-plug}
% \1 Installation \urlref{https://github.com/junegunn/vim-plug#unix-linux}{Unix, Linux}
%   \2 \verb|sh -c 'curl -fLo "${XDG_DATA_HOME:-$HOME/.local/share}"/nvim/site/autoload/plug.vim --create-dirs \|
%   \verb|https://raw.githubusercontent.com/junegunn/vim-plug/master/plug.vim'|
%   \2 Reload \verb|.vimrc| and \verb|:PlugInstall| to install plugins.
% \end{myitemize}
% ^^A ]]] End of subsubsection `junegunn/vim-plug'.
%
% \subsubsection{Shougo/dein.vim}^^A [[[
% \begin{myitemize}
% \1 Dark powered Vim/Neovim plugin manager \vimhelp{dein.txt}
% \1 一人で開発しているのか,deoplete,deoppetなどの開発が非常に遅い
% \1 OSをインストールし直す毎にエラーやpluginが上手く機能しない事態に陥る
% \1 
% \end{myitemize}
% ^^A ]]] End of subsubsection `Shougo/dein.vim'.
%
% \subsubsection{Conclusion}^^A [[[
% \begin{myitemize}
% \1 junegunn/vim-plug
%   \2 記述が簡単
%   \2 初期設定の速度が速い
%   \2 Shougo/dein.vimは記述が複雑
% \end{myitemize}
% ^^A ]]] End of subsubsection `Conclusion'.
%
% ^^A ]]] End of subsection `Plugin manager'.
%
% \subsection{File explorer}^^A [[[
%
% \begin{concealableitemize}^^A [[[
% \1 The origin of `file explorer'
%   \2 A powerful file explorer implemented in Vim script
%   \urlref{https://github.com/Shougo/vimfiler.vim#vimfiler}{vimfiler.vim}
%   \2 The NERDTree is a file system explorer for the Vim editor
%   \urlref{https://github.com/preservim/nerdtree#introduction}{nerdtree}.
% \end{concealableitemize}^^A ]]]
%
% \subsubsection{関連pluginについて比較・言及されている記事}^^A [[[
% \begin{myitemize}
% \1 \href{https://vim.blue/compare-filer-plugins/}{どれが良い?Vimのファイラ系プラグインを比較してみた}
% \1 
% \end{myitemize}
% ^^A ]]] End of subsubsection `***'.
%
% \subsubsection{preservim/nerdtree}^^A [[[
% \begin{myitemize}
% \1 Homepage: \href{https://github.com/preservim/nerdtree}{preservim/nerdtree}
%   \2 
% \1 \verb|mo|でpdfなどを開ける
%   \2 \href{https://vi.stackexchange.com/questions/19657/execute-a-file-from-nerdtree}{mo}
% \1 nerdtreeのbookmarkは\verb|~/.NERDTreeBookmarks|に保存されている
% \1 
% \end{myitemize}
% ^^A ]]] End of subsubsection `preservim/nerdtree'.
%
% \subsubsection{Shougo/vimfiler.vim}^^A [[[
% \begin{myitemize}
% \1 Homepage: \href{https://github.com/Shougo/vimfiler.vim}{Shougo/vimfiler.vim}
%   \2 実際に使ってみたが,bookmarkがunite.vimを経由して使うので面倒
%   \2 vimfilerで\verb|x|でシステムの関連付けを実行
%     \3 例えば,\verb|.pdf|で\verb|x|を実行すると\verb|evince|が起動してpdfが見れる
%     \urlref{http://i05nagai.github.io/memorandum/vim/vimfiler.html}{vimfiler}
%   \2 vimfilerでファイルのbookmark
%     \3 \url{http://i05nagai.github.io/memorandum/vim/vimfiler.html}
%     \3 \url{http://baqamore.hatenablog.com/entry/2015/06/24/213320}
%       \4 \verb|~/.cache/unite/bookmark/default|にbookmarkが保存されている
% \end{myitemize}
%
% vimfilerはunite.vimを必要とする\urlref{https://github.com/Shougo/vimfiler.vim}{vimfiler.vim}.
% 現時点ではvimfilerはdenite.nvimには対応していないので,unite.vimを使うことにする
% (cf. \url{https://github.com/Shougo/vimfiler.vim/issues/379}).
% 対応し次第,denite.nvimに移行するつもり.
%
% \verb|vimfiler_enable_auto_cd|について調べよ.同じくunitでlcdについて調べよ.
%
% \begin{myitemize}
% \1 \viminline{let g:vimfiler_as_default_explorer = 1}
%   \2 If this variable is true, Vim use vimfiler as file manager instead of netrw (cf. vimfiler.txt, VARIABLES).
% \1 \viminline{:VimFiler -force-quit -winwidth=30 ...}
%   \2 \href{http://lsifrontend.blog100.fc2.com/blog-entry-343.html}{vimfilerのoption}
%   \2 \verb|-force-quit|
%     \3 Exit the vimfiler buffer after firing an action. \verb|:h vimfiler-options-force-quit|
%   \2 \verb|-winwidth|
%     \3 Specifies the width of a vimfiler buffer.
%   \2 \verb|-simple|
%     \3 Enable vimfiler simple mode.
%   \2 \verb|-split|
%     \3 Split vimfiler buffer.
% \1 \viminline{let g:vimfiler_edit_action = ...}
%   \2 \href{http://lsifrontend.blog100.fc2.com/blog-entry-343.html}{vimfiler edit action}
% \1 \viminline{let g:vimfiler_ignore_pattern = ...}
%   \2 Specify the regexp pattern string or list to ignore candidates of the source.
%   \2 \href{http://qiita.com/termoshtt/items/3cf7596a1c81c0a4c160}{vimfiler ignore pattern}
%   \2 \href{https://github.com/Shougo/vimfiler.vim/issues/120}{vimfiler ignore pattern}
%   \2 \vimhelp{vimfiler}にあるように\verb|['^\.pdf']|としてもダメだった
% \1 \viminline{vimfiler#do_switch_action}
%   \2 \href{http://baqamore.hatenablog.com/entry/2016/02/13/062555}{ref}
% \1 \viminline{vimfiler#set_execute_file()}
%   \2 \href{http://i05nagai.github.io/memorandum/vim/vimfiler.html}{ref}
% \end{myitemize}
%
%    \begin{macrocode}
% [[plugins]]
% repo = 'Shougo/vimfiler'
% hook_add = '''
% let g:vimfiler_as_default_explorer = 1
% nnoremap <silent>f :VimFiler -force-quit -winwidth=30 -simple -split<CR>
% " nnoremap <silent>f :VimFiler -force-quit -winwidth=30 -simple -split -direction='topright'<CR>
% let g:vimfiler_edit_action = 'vsplit'
% " let g:vimfiler_edit_action = 'split'
% " let g:vimfiler_edit_action = 'tabopen'
% let g:vimfiler_ignore_pattern = '\%(.png\|.pyc\)$'
% % ^^A  " let g:vimfiler_ignore_pattern = '\(^\.\|\~$\|\.pdf$\|\.[oad]$\)'
% % ^^A  " let g:vimfiler_ignore_pattern = '\%(.pdf\|.png\|.pyc\)$'
% nnoremap s vimfiler#do_switch_action('split')
% nnoremap v vimfiler#do_switch_action('vsplit')
% % ^^A call vimfiler#set_execute_file('pdf', 'evince')
% '''
%    \end{macrocode}
% ^^A ]]] End of subsubsection `Shougo/vimfiler.vim'.
%
% \subsubsection{Conclusion}^^A [[[
% \begin{myitemize}
% \1 preservim/nerdtree
%   \2 bookmarksが簡単
%   \2 Shougo/vimfiler.vimはbookmarksの設定方法がよくわからない
% \end{myitemize}
% ^^A ]]] End of subsubsection `Conclusion'.
%
% ^^A ]]] End of subsection `File explorer'.
%
% \subsection{Commenting}^^A [[[
% \begin{myitemize}
% \1 SpaceVimがnerdcommenterを使っている\urlref{https://spacevim.org/documentation/#commenting}{Commenting}
% \1 
% \1 
% \end{myitemize}
%
%
% ^^A ]]] End of subsection `Commenting'.
%
% \subsection{display git diff}^^A [[[
% \begin{myitemize}
% \1 airblade/vim-gitgutter
% \1 'mhinz/vim-signify'
% \end{myitemize}
%
%
%
%
% ^^A ]]] End of subsection `***'.
%
% \subsection{auto close bracket}^^A [[[
% \begin{myitemize}
% \1 'Townk/vim-autoclose'
% \1 
% \end{myitemize}
%
%
% ^^A ]]] End of subsection `auto close bracket'.
%
% \subsection{auto completion}^^A [[[
% \begin{myitemize}
% \1 \href{https://www.slant.co/topics/3999/~best-semantic-autocompletion-plugins-for-vim}{What are the best semantic autocompletion plugins for Vim?}
% \1 \href{https://github.com/ycm-core/YouCompleteMe/issues/1751}{Can I install YCM with vim-plug?}
% \1 \href{https://github.com/ycm-core/YouCompleteMe}{ycm-core/YouCompleteMe}
% \1 YouCompleteMeはインストールの仕方が悪いのか,エラー発生して上手くいかない
% \1 vim built in function???
%   \2 In insert mode, press \verb|Ctrl+n| or \verb|Ctrl+n| to complete word
% \end{myitemize}
%
% \subsubsection{Shougo/ddc.vim}^^A [[[
% \begin{myitemize}
% \1 \href{https://github.com/Shougo/ddc.vim}{Shougo/ddc.vim - GitHub}
%   \2 Dark deno-powered completion framework for neovim/Vim8
% \1
% \end{myitemize}
%
%
%
%
% ^^A ]]] End of subsubsection `Shougo/ddc.vim'.
%
% ^^A ]]] End of subsection `auto completion'.
%
% \subsection{Snippets}^^A [[[
% \begin{myitemize}
% \1 \href{https://www.vim.org/scripts/script.php?script_id=3633}{vim-snippets : snippets for a variety of vim plugins}
% \1 \href{https://zenn.dev/shougo/articles/snippet-plugins-2020}{スニペットプラグインについて 2020 年版}
% \1 \href{https://github.com/SirVer/ultisnips}{SirVer/ultisnips}
% \end{myitemize}
%
% \subsubsection{Shougo/deoppet}^^A [[[
% \begin{myitemize}
% \1 The neo-snippet plugin contains snippet source
% \1 \vimhelp{deoppet-examples}
% \end{myitemize}
%
% \begin{vimcode}
% Plug 'Shougo/deoppet.nvim'
% imap <C-k>  <Plug>(deoppet_expand)
% imap <C-f>  <Plug>(deoppet_jump_forward)
% imap <C-b>  <Plug>(deoppet_jump_backward)
% smap <C-f>  <Plug>(deoppet_jump_forward)
% smap <C-b>  <Plug>(deoppet_jump_backward)
%
% call deoppet#initialize()
% call deoppet#custom#option('snippets',
%\ map(globpath(&runtimepath, 'neosnippets', 1, 1),
%\     { _, val -> { 'path': val } }))
%
% " Use deoppet source.
%call ddc#custom#patch_global('sources', ['deoppet'])
%
% " Change source options
%call ddc#custom#patch_global('sourceOptions', {
%      \ 'deoppet': {'dup': v:true, 'mark': 'dp'},
%      \ })
% \end{vimcode}
% ^^A ]]] End of subsubsection `Shougo/deoppet'.
%
% \subsubsection{Shougo/neosnippet}^^A [[[
% \begin{myitemize}
% \1 使用してみての所感
%   \2 snippet挿入時indentが崩れる
%     \3 Python, YAMLだと致命的
%   \1 snippet挿入時,挙動が変になることがある
% \1 
% \end{myitemize}
%
% \begin{myitemize}
% \1 The neo-snippet plugin contains snippet source
% \1 \viminline{let g:neosnippet#snippets_directory = ...}
%   \2 This variable appoints a path to user-defined snippet files
%   \vimhelp{g:neosnippet#snippets_directory}.
% \1 \viminline{<Plug>(neosnippet_expand_or_jump)}
%   \2 Expand a snippet in current cursor position
%   \vimhelp{<Plug>(neosnippet_expand)}.
% \1 基本構造
%   \2 \vimhelp{neosnippet-examples}
% \end{myitemize}
%
% \begin{vimcode}
% Plug 'Shougo/neosnippet'
% let g:neosnippet#snippets_directory = '~/Dropbox/configuration-files/vim/output-files/snippets'
% imap <C-k> <Plug>(neosnippet_expand_or_jump)<Esc>:only<CR>li
% smap <C-k> <Plug>(neosnippet_expand_or_jump)
% xmap <C-k> <Plug>(neosnippet_expand_target)
%
%
%
% " SuperTab like snippets' behavior.
% "imap <expr><TAB>
% " \ pumvisible() ? "\<C-n>" :
% " \ neosnippet#expandable_or_jumpable() ?
% " \    "\<Plug>(neosnippet_expand_or_jump)" : "\<TAB>"
% "smap <expr><TAB> neosnippet#expandable_or_jumpable() ?
% " \ "\<Plug>(neosnippet_expand_or_jump)" : "\<TAB>"
%
% " Enable snipMate compatibility feature.
% " let g:neosnippet#enable_snipmate_compatibility = 1
% \end{vimcode}
% ^^A ]]] End of subsubsection `Shougo/neosnippet'.
%
% \subsubsection{Shougo/neosnippet-snippets}^^A [[[
% \begin{vimcode}
% Plug 'Shougo/neosnippet-snippets'
% \end{vimcode}
% ^^A ]]] End of subsubsection `Shougo/neosnippet-snippets'.
%
% \subsubsection{SirVer/ultisnips}^^A [[[
% \begin{myitemize}
% \1 \href{https://github.com/SirVer/ultisnips}{SirVer/ultisnips}
% \1 
% \1 
% \end{myitemize}
% ^^A ]]] End of subsubsection `SirVer/ultisnips'.
%
% ^^A ]]] End of subsection `Snippets'.
%
% ^^A ]]] End of section `Comparison of plugins for Neovim'.
%
% ^^A End of file `configuration-files.dtx'.
