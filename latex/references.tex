\documentclass[
	programming,
	short-document,
	show-notes,
]{mydocument}
\begin{document}
\mytitle{References notes}

\begin{abstract}% [[[
自分用の参考文献メモ
\end{abstract}% ]]]

\begin{mytip}% [[[
\begin{myitemize}
\1 \texdoc{biblatex-cheatsheet}
\1 
\1 
\end{myitemize}
\end{mytip}% ]]]

\mytableofcontents

\section{\texttt{references.bib}}% [[[

\subsection{Template}% [[[
\begin{texcode}
@type{key,
  author = "foo",
  title = "bar",
  year = 2000, }
\end{texcode}
% ]]] End of subsection `Template'.

\subsection{Basic code ???}% [[[
\begin{myitemize}
\1 \mycite[page=189]{LaTeX2e美文書作成入門}
\1 \mycite[page=]{}
\1 
\end{myitemize}
% ]]] End of subsection `Basic code'.

\subsection{type}% [[[
\begin{myitemize}
\1 The origin of `type'
	\2 \texdoc[section=3.1 Entry Types]{bibtex}
	\2 \texdoc{biblatex-cheatsheet}
\1 
\end{myitemize}
% ]]] End of subsection `type'.

\subsection{key}% [[[
\begin{myitemize}
\1 The origin of `key'
	\2 \texdoc[section=3.2 Fields]{bibtex}の\verb|key|
	\2 This environment prints the actual bibliography,
	and the \verb|\bibitem| commands link the entries to the citations via the key, here \verb|jon90|.
	\texdoc[section=1 Introduction]{natbib}
\1 基本的にはタイトルをそのまま\verb|key|にする
	\2 著者名は含めない.もし\verb|key|を著者名から始めると著者名を覚えている必要があるし,
	複数人の書籍やWebサイトの場合,困るから.
		\3 同じタイトルのものがある場合,著者名を入れる感じか?
	\2 日本語タイトルの場合もそのまま日本語で\verb|key|にする
		\3 英訳にすることが難しいから
		\3 ローマ字表記にはしない.日本語をローマ字表記したものを\verb|key|にすると,
		「おう」\myarrow`\verb|o|', `\verb|ou|', `\verb|oh|'などがあり混乱を招くから
\1 大文字と小文字の違いしかない参照は用いてはならない\mycite[page=352]{teach-yourself-latex2e}.
	\2 特別な理由がない限り小文字を使用する.理由はいちいちSHIFTキーを押すのが面倒だから
\1 なお,\verb|key|はVimでは\verb|Ctrl+x, Ctrl+k|で補完できる
	\2 日本語でも問題なく補完できる
\1 区切り文字は\doublequotes{\texttt{-}}\texdoc[section=2.1 New BIB\TeX features]{bibtex}
\end{myitemize}
% ]]] End of subsection `key'.

\subsection{author}[updated=2022-05-05T21:17:34]% [[[
\begin{myitemize}
\1 \verb|author = "北原 康貴"|のように,姓と名の間に半角空白を入れること\mycite[page=361]{独習LaTeX2e}
\end{myitemize}
% ]]] End of subsection `author'.

\subsection{note}% [[[
\begin{myitemize}
\1 \verb|note = "~~~"|は著者の出身大学を記入することにする
\1 
\end{myitemize}
% ]]] End of subsection `note'.

\subsection{title}% [[[
\begin{myitemize}
\1 bookの場合:裏表紙のところを使用する
\1 websiteの場合:\fbox{F12}でhtml表示したときの\verb|<title> *** </title>|を使用する
\end{myitemize}
% ]]] End of subsection `title'.

\subsection{url}% [[[
urlの箇所は次のようにする.
\begin{verbatim}
url = "http://...",
\end{verbatim}
hyperrefパッケージの\verb|\url|を使って\verb|url = "\url{http://...}"|としなくても,自動的にハイパーリンクしてくれる.
% ]]] End of subsection `url'.

\subsection{year}[updated=2022-05-05T19:40:57]% [[[
\begin{myitemize}
\1 数字は\verb|"|で囲まない
\mycite[page=189]{LaTeX2e美文書作成入門},
\mycite[page=357]{独習LaTeX2e}
\end{myitemize}
% ]]] End of subsection `year'.

% ]]] End of section `references.bib'.

\section{\texttt{references.yaml}}% [[[
\begin{yamlcode}
category:

- title: Book
  key: bar
  author:
  - A
  - B
  publisher: ***
  year: 2000
  note:
  - ***
  type: book
  updated: 2022-05-05T18:11:23

- title: Website
  key: ***
  type: misc
  url: https:***
  updated: 2022-05-05T19:28:17

\end{yamlcode}
% ]]] End of section `references.yaml'.

\end{document}

%
% \part{\myverb{myreferences.py}}^^A [[[
%
% \section{Science \& Math}^^A [[[
% \doublequotes{Science \& Math}の由来は
% \href{https://www.amazon.com/books-used-books-textbooks/b?ie=UTF8&node=283155}{Amazon.com: Books}より.
%
% \subsection{Math}^^A [[[
%
% \subsubsection{Analysis}^^A [[[
%
% \begin{macro}{kaisekinyumon-jo}^^A [[[
% \begin{pythoncode}
%<*myreferences.py>
add_ref(
    type='book',
    key='kaisekinyumon-jo',
    author='松坂 和夫',
    title='解析入門 上',
    publisher='岩波書店',
    year='2018',
    note='東京大学理学部数学科卒業',)
%</myreferences.py>
%<*reference>
@book{kaisekinyumon-jo,
	author = "松坂和夫",
	title = "解析入門 上",
	publisher = "岩波書店",
	year = 2018,
	note = "東京大学理学部数学科卒業",
}
%</reference>
% \end{pythoncode}
% \end{macro}^^A ]]]
%
% \begin{macro}{kaisekinyumon-chu}^^A [[[
%    \begin{macrocode}
%<*myreferences.py>
add_ref(
    type='book',
    key='kaisekinyumon-chu',
    author='松坂 和夫',
    title='解析入門 中',
    publisher='岩波書店',
    year='2018',
    note='東京大学理学部数学科卒業',)
%</myreferences.py>
%<*reference>
@book{kaisekinyumon-chu,
	author = "松坂和夫",
	title = "解析入門 中",
	publisher = "岩波書店",
	year = 2018,
	note = "東京大学理学部数学科卒業",
}
%</reference>
%    \end{macrocode}
% \end{macro}^^A ]]]
%
% ^^A ]]] End of subsubsection `Analysis'.
%
% \begin{macro}{principles-of-mathematical-analysis}^^A [[[
% \href{https://en.wikipedia.org/wiki/Walter_Rudin}{Wikipedia}によると,
% この本は世界規模で数学教育に影響を与えているらしい.
%    \begin{macrocode}
%<*reference>
@book{principles-of-mathematical-analysis,
	author = "Walter Rudin",
	title = "Principle of Mathematical Analysis",
	edition = "3",
}
%</reference>
%    \end{macrocode}
% \end{macro}^^A ]]]
%
% \subsubsection{数学A}^^A [[[
% \begin{pythoncode}
%<*myreferences.py>
add_ref(
    type='book',
    key='数学A',
    title='数学A',
    publisher='数研出版',
    year='2006',
    )
%</myreferences.py>
% \end{pythoncode}
% ^^A ]]] End of subsubsection `数学A'.
%
% \subsubsection{数学B}^^A [[[
% \begin{pythoncode}
%<*myreferences.py>
add_ref(
    type='book',
    key='数学B',
    title='数学B',
    publisher='数研出版',
    year='2007',
    )
%</myreferences.py>
% \end{pythoncode}
% ^^A ]]] End of subsubsection `数学B'.
%
% \subsubsection{数学C}^^A [[[
% \begin{pythoncode}
%<*myreferences.py>
add_ref(
    type='book',
    key='数学C',
    title='数学C',
    publisher='数研出版',
    year='2007',
    )
%</myreferences.py>
% \end{pythoncode}
% ^^A ]]] End of subsubsection `数学C'.
%
% ^^A ]]] End of subsection `Math'.
%
% \subsection{Science}^^A [[[
%
% \subsubsection{Classical mechanics}^^A [[[
% \myemph{Classical Mechanics} is a textbook about that subject written by Herbert Goldstein,
% a professor at Columbia University.
% Intended for advanced undergraduate and beginning graduate students,
% it has been one of the standard references in its subject around the world
% since its first publication in 1951 \mycite[keyword=Classical Mechanics (Goldstein)]{wikipedia}.
% \begin{pythoncode}
%<*myreferences.py>
add_ref(
    type='book',
    key='classical-mechanics',
    author='Herbert Goldstein',
    title='Classical Mechanics',
    publisher='Addison-Wesley',
    edition='3', )
%</myreferences.py>
% \end{macrocode}
% ^^A ]]] End of subsubsection `Classical mechanics'.
%
%
% \subsubsection{Quantum mechanics}^^A [[[
% \myemph{Quantum Computation and Quantum Information} is a textbook about quantum information science
% written by Michael Nielsen and Isaac Chuang, regarded as a standard text on the subject
% \mycite[keyword=Quantum Computation and Quantum Information]{wikipedia}
% \begin{pythoncode}
%<*myreferences.py>
add_ref(
    type='book',
    key='quantum-computation-and-quantum-information',
    author='Michael A. Nielsen \& Isaac L. Chuang',
    title='Quantum Computation and Quantum Information',
    publisher='Cambridge University Press',
    edition='10', )
%</myreferences.py>
% \end{macrocode}
% ^^A ]]] End of subsubsection `Quantum mechanics'.
%
%
%
%
% ^^A ]]] End of subsection `Science'.
%
%
%
%
%
%
% ^^A ]]] End of section `Science & Math'.
%
% \section{Computers \& Technology}^^A [[[
%
% \begin{concealableitemize}^^A [[[
% \1 \doublequotes{Computers \& Technology}の由来
%   \2 \urlref{https://www.amazon.com/books-used-books-textbooks/b?ie=UTF8&node=283155}{Amazon.com: Books}
% \end{concealableitemize}^^A ]]]
%
% \subsection{Bash}^^A [[[
%
% \subsubsection{learning-the-bash-shell}^^A [[[
% \begin{pythoncode}
%<*myreferences.py>
add_ref(
    type='book',
    key='learning-the-bash-shell',
    author='Cameron Newham and Bill Rosenblatt',
    title='Learning the bash Shell',
    publisher="O'Reilly",
    year='2005', )
%</myreferences.py>
% \end{pythoncode}
% ^^A ]]] End of subsubsection `learning-the-bash-shell'.
%
% ^^A ]]] End of subsection `Bash'.
%
% \subsection{C/C++}^^A [[[
%
% \begin{macro}{pep-7-style-guide-for-c-code'}^^A [[[
% \begin{pythoncode}
%<*myreferences.py>
add_ref(
    type='misc',
    key='pep-7-style-guide-for-c-code',
    title='PEP 7 -- Style Guide for C Code',
    url='https://www.python.org/dev/peps/pep-0007/',
    )
%</myreferences.py>
% \end{pythoncode}
% \end{macro}^^A ]]]
%
% \begin{macro}{teach-yourself-c}^^A [[[
%    \begin{macrocode}
%<*myreferences.bib>
@book{teach-yourself-c,
	author = "ハーバート・シルト",
	title = "独習C",
	publisher = "翔泳社",
	year = 2014,
}
%</myreferences.bib>
%    \end{macrocode}
% \end{macro}^^A ]]]
%
% \begin{macro}{teach-yourself-cpp}^^A [[[
% \doublequotes{c++}とすると補完しないので\doublequotes{cpp}とした.
%    \begin{macrocode}
%<*reference>
@book{teach-yourself-cpp,
	author = "ハーバート・シルト",
	title = "独習C++",
	publisher = "翔泳社",
	year = 2014,
}
%</reference>
%    \end{macrocode}
% \end{macro}^^A ]]]
%
% ^^A ]]] End of subsection `C/C++'.
%
% \subsection{Git}^^A [[[
%
%
% \begin{macro}{version-control-with-git}^^A [[[
%    \begin{macrocode}
%<*myreferences.py>
add_ref(
    type='book',
    key='version-control-with-git',
    author='Jon Loeliger and Matthew McCullough',
    title='Version Control with Git',
    publisher="O'Reilly",
    year='2012', )
%</myreferences.py>
%    \end{macrocode}
%
%
%
%    \begin{macrocode}
%<*myreferences.bib>
@book{version-control-with-git,
    author = "Jon Loeliger and Matthew McCullough",
    title = "Version Control with Git",
    publisher = "O'Reilly",
    year = 2012,
}
%</myreferences.bib>
%    \end{macrocode}
% \end{macro}^^A ]]]
%
% ^^A ]]] End of subsection `Git'.
%
% \subsection{Python}^^A [[[
%
% \begin{macro}{pep-8-style-guide-for-python-code'}^^A [[[
% \begin{pythoncode}
%<*myreferences.py>
add_ref(
    type='misc',
    key='pep-8-style-guide-for-python-code',
    title='PEP 8 -- Style Guide for Python Code',
    url='https://www.python.org/dev/peps/pep-0008/',
    )
%</myreferences.py>
% \end{pythoncode}
% \end{macro}^^A ]]]
%
% \begin{macro}{introducting-python}^^A [[[
%    \begin{macrocode}
%<*myreferences.py>
add_ref(
    type='book',
    key='introducting-python',
    author='Bill Lubanovic',
    title='Introducting Python',
    publisher="O'Reilly",
    year='2014',
)
%</myreferences.py>
%    \end{macrocode}
%
%
%
%    \begin{macrocode}
%<*reference>
@book{introducting-python,
	author = "Bill Lubanovic",
	title = "Introducting Python",
	publisher = "O'Reilly",
	year = 2014 }
%</reference>
%    \end{macrocode}
% \end{macro}^^A ]]]
%
% \begin{macro}{learning-python}^^A [[[
%    \begin{macrocode}
%<*reference>
@book{learning-python,
	author = "Mark Lutz",
	title = "Learning Python",
	publisher = "O'Reilly",
	year = 2013,
	edition = "7", }
%</reference>
%    \end{macrocode}
% \end{macro}^^A ]]]
%
% \begin{macro}{programming-python}^^A [[[
%    \begin{macrocode}
%<*reference>
@book{programming-python,
	author = "Mark Lutz",
	title = "Programming Python",
	publisher = "O'Reilly",
	year = 2010,
	edition = "4", }
%</reference>
%    \end{macrocode}
% \end{macro}^^A ]]]
%
% \begin{macro}{python-documentation}^^A [[[
%    \begin{macrocode}
%<*reference>
@misc{python-documentation,
	title = "Python documentation",
	url = "https://docs.python.org/",
	howpublished = "URL: \url{https://docs.python.org/}", }% この行は一時的
%</reference>
%    \end{macrocode}
% \end{macro}^^A ]]]
%
% \begin{macro}{automate-the-boring-stuff-with-python}^^A [[[
%    \begin{macrocode}
%<*reference>
@book{automate-the-boring-stuff-with-python,
	author = "Al Sweigart",
	title = "AUTOMATE THE BORING STUFF WITH PYTHON",
	publisher = "no starch press",
	year = 2015 }
%</reference>
%    \end{macrocode}
% \end{macro}^^A ]]]
%
% ^^A ]]] End of subsection `Python'.
%
% \subsection{Machine learning}^^A [[[
%
% \subsubsection{Understanding Machine Learning}^^A [[[
% \begin{pythoncode}
%<*myreferences.py>
add_ref(
    type='book',
    key='understanding-machine-learning-from-theory-to-algorithms',
    author='Shai Shalev-Shwartz, Shai Ben-David',
    title='Understanding Machine Learning From Theory To Algorithms',
    publisher='Cambridge University Press',
    year='2014', )
%</myreferences.py>
% \end{pythoncode}
% ^^A ]]] End of subsubsection `Understanding Machine Learning'.
%
% \subsubsection{deep-learning-from-scratch}[updated=]^^A [[[
% \begin{myitemize}
% \1 The origin of `\verb|deep-learning-from-scratch|'
%   \2 \url{https://github.com/oreilly-japan/deep-learning-from-scratch}
%   \mycite[page=vii]{deep-learning-from-scratch}
% \end{myitemize}
%
% \begin{pythoncode}
%<*myreferences.py>
add_ref(
    type='book',
    key='deep-learning-from-scratch',
    author='斎藤 康毅',
    title='ゼロから作るDeep Learning --- Pythonで学ぶディープラーニングの理論と実装',
    publisher='オライリー・ジャパン',
    year='2020',
    note='東京工業大学工学部卒,東京大学大学院学際情報学府修士課程修了', )
%</myreferences.py>
% \end{pythoncode}
% ^^A ]]] End of subsubsection `deep-learning-from-scratch'.
%
% ^^A ]]] End of subsection `Machine learning'.
%
% \subsection{\TeX}^^A [[[
%
% \begin{macro}{latexmk}^^A [[[
%    \begin{macrocode}
%<*reference>
@misc{latexmk,
    title = "LATEXMK",
    howpublished = "\href{run:/usr/share/doc/latexmk/latexmk.pdf}{\texttt{latexmk.pdf}}", }
%</reference>
%    \end{macrocode}
% \end{macro}^^A ]]]
%
% ^^A ]]] End of subsection `\TeX'.
%
% \subsection{Windows}^^A [[[
%
%
%
% \subsubsection{Excel}^^A [[[
%
% \begin{macro}{excel-function-zenjiten}^^A [[[
%
% \begin{pythoncode}
%<*myreferences.py>
add_ref(
    type='book',
    key='excel-function-zenjiten',
    author='羽山博 吉川明広',
    title='できるポケットExcel関数全辞典',
    publisher='株式会社インプレス',
    year='2015', )
%</myreferences.py>
% \end{pythoncode}
% \end{macro}^^A ]]]
%
% \begin{macro}{excel-vba-programming-for-dummies}^^A [[[
% \begin{pythoncode}
%<*myreferences.py>
add_ref(
  type='book',
  key='excel-vba-programming-for-dummies',
  author='John Walkenbach',
  title='Excel VBA Programming For Dummies',
  publisher='John Wiley \& Sons, Inc.',
  year='2013',
  edition='3', )
%</myreferences.py>
% \end{pythoncode}
% \end{macro}^^A ]]]
%
% ^^A ]]] End of subsubsection `Excel'.
%
% ^^A ]]] End of subsection `Windows'.
%
% \subsection{etc.}^^A [[[
%
% \subsubsection{using-docker}^^A [[[
% \begin{pythoncode}
%<*myreferences.py>
add_ref(
    type='book',
    key='using-docker',
    author='Adrian Mouat',
    title='Using Docker',
    publisher="O'Reilly",
    year='2016',)
%</myreferences.py>
% \end{pythoncode}
% ^^A ]]] End of subsubsection `using-docker'.
%
% \subsubsection{pc-jisaku-tune-up-toranomaki-2020}^^A [[[
% 以下の情報は\mycite[page=512]{pc-jisaku-tune-up-toranomaki-2020}
% \begin{pythoncode}
%<*myreferences.py>
add_ref(
    type='book',
    key='pc-jisaku-tune-up-toranomaki-2020',
    author='小川 亨',
    title='PC自作・チューンナップ虎の巻二〇二〇',
    publisher='株式会社インプレス',
    year='2019',)
%</myreferences.py>
% \end{pythoncode}
% ^^A ]]] End of subsubsection `pc-jisaku-tune-up-toranomaki-2020'.
%
% ^^A ]]] End of subsection `etc.'.
%
% ^^A ]]] End of section `Computers & Technology'.
%
% \section{English}^^A [[[
%
% \begin{concealablenote}^^A [[[
% \doublequotes{English}の由来は自分で決めた(特に変更する必要はないと思う)
% \end{concealablenote}^^A ]]]
%
% \subsection{Dictionary}^^A [[[
% \href{https://en.wikipedia.org/wiki/Comparison_of_English_dictionaries}{ここ}に各辞書が比較さてれいる.
%
% \subsubsection{lexico}^^A [[[
% \href{https://www.lexico.com/about}{About}
% \begin{pythoncode}
%<*myreferences.py>
add_ref(
    type='misc',
    key='lexico',
    title='LEXICO',
    note='powered by OXFORD',
    url='https://www.lexico.com/',)
%</myreferences.py>
% \end{pythoncode}
% ^^A ]]] End of subsubsection `lexico'.
%
% \subsubsection{oxford-learners-dictionaries}^^A [[[
% \begin{pythoncode}
%<*myreferences.py>
add_ref(
    type='misc',
    key='oxford-learners-dictionaries',
    title="Oxford Learner's Dictionaries",# double quotesにしないとエラー
    url='https://www.oxfordlearnersdictionaries.com/',)
%</myreferences.py>
% \end{pythoncode}
% ^^A ]]] End of subsubsection `oxford-learners-dictionaries'.
%
% \subsubsection{longman}^^A [[[
% \href{https://www.ldoceonline.com/about.html}{About}
% \begin{pythoncode}
%<*myreferences.py>
add_ref(
    type='misc',
    key='longman',
    title='Longman Dictionary of Contemporary English',
    url='http://www.ldoceonline.com/',)
%</myreferences.py>
% \end{pythoncode}
% ^^A ]]] End of subsubsection `longman'.
%
% \subsubsection{cambridge-dictionary}^^A [[[
% \begin{pythoncode}
%<*myreferences.py>
add_ref(
    type='misc',
    key='cambridge-dictionary',
    title='Cambridge Dictionary',
    url='https://dictionary.cambridge.org/',)
%</myreferences.py>
% \end{pythoncode}
% ^^A ]]] End of subsubsection `cambridge-dictionary'.
%
% \subsubsection{genius}^^A [[[
% \begin{pythoncode}
%<*myreferences.py>
add_ref(
    type='book',
    key='genius',
    author='小西 友七',
    title='ジーニアス英和辞典',
    publisher='大修館書店',
    year='2001', )
%</myreferences.py>
% \end{pythoncode}
% ^^A ]]] End of subsubsection `genius'.
%
% \begin{macro}{newbury-house-dictionary}^^A [[[
%    \begin{macrocode}
%<*reference>
@book{newbury-house-dictionary,
	author = "Philip M. Rideout",
	title = "Heinle's Newbury House Dictionary of American English",
	edition = "4",
}
%</reference>
%    \end{macrocode}
% \end{macro}^^A ]]]
%
% \subsubsection{weblio}^^A [[[
% \begin{pythoncode}
%<*myreferences.py>
add_ref(
    type='misc',
    key='weblio',
    title='weblio 英和辞典・和英辞典',
    url='https://ejje.weblio.jp/',)
%</myreferences.py>
% \end{pythoncode}
% ^^A ]]] End of subsubsection `weblio'.
%
%
%
% ^^A ]]] End of subsection `Dictionary'.
%
% \subsection{Grammar}^^A [[[
%
% \subsubsection{ロイヤル英文法}^^A [[[
% \begin{pythoncode}
%<*myreferences.py>
add_ref(
    type='book',
    key='royal-english-grammar',
    author='綿貫 陽,宮川 幸久',
    title='徹底例解ロイヤル英文法',
    publisher='旺文社', )
%</myreferences.py>
% \end{pythoncode}
% ^^A ]]] End of subsubsection `ロイヤル英文法'.
%
% \begin{macro}{maintop}^^A [[[
%    \begin{macrocode}
%<*reference>
@book{maintop,
	author = "生田 友七",
	title = "MAINTOP総合英語",
	publisher = "山口書店",
	year = 2007,
	note = "南山大学外国語学部英米学科卒業", }
%</reference>
%    \end{macrocode}
% \end{macro}^^A ]]]
%
%
%
%
% ^^A ]]] End of subsection `Grammar'.
%
% \subsection{Wordbook}^^A [[[
%
% \begin{concealablenote}^^A [[[
% \doublequotes{wordbook} \urlref{https://eikaiwa.dmm.com/uknow/questions/63183/}{単語帳って英語でなんて言うの?}
% \end{concealablenote}^^A ]]]
%
% \subsubsection{単語王}^^A [[[
% \begin{pythoncode}
%<*myreferences.py>
add_ref(
    type='book',
    key='tangoou-2202',
    author='中澤 一',
    title='単語王2202',
    publisher='株式会社 オー・メソッド出版', )
%</myreferences.py>
% \end{pythoncode}
% ^^A ]]] End of subsubsection `単語王'.
%
%
%
%
%
% ^^A ]]] End of subsection `Wordbook'.
%
% ^^A ]]] End of section `English'.
%
% \section{Business}^^A [[[
% \doublequotes{Business}の由来はAmazon.com: Booksより.
%
% \subsection{businessdou-life}^^A [[[
% \begin{pythoncode}
%<*myreferences.py>
add_ref(
    type='book',
    key='businessdou-life',
    author='内山 早苗',
    title='{\\small 新社会人基本}Newビジネス道PLUS「体」',
    publisher='株式会社 日本能率協会マネジメントセンター',
    year='2016',)
%</myreferences.py>
% \end{pythoncode}
% ^^A ]]] End of subsection `businessdou-life'.
%
% \subsection{businessdou-mind}^^A [[[
% \begin{pythoncode}
%<*myreferences.py>
add_ref(
    type='book',
    key='businessdou-mind',
    author='内山 早苗',
    title='{\\small 新社会人基本}Newビジネス道PLUS「心」',
    publisher='株式会社 日本能率協会マネジメントセンター',
    year='2016',)
%</myreferences.py>
% \end{pythoncode}
% ^^A ]]] End of subsection `businessdou-mind'.
%
% \subsection{businessdou-skill}^^A [[[
% \begin{pythoncode}
%<*myreferences.py>
add_ref(
    type='book',
    key='businessdou-skill',
    author='内山 早苗',
    title='{\\small 新社会人基本}Newビジネス道PLUS「技」',
    publisher='株式会社 日本能率協会マネジメントセンター',
    year='2016',)
%</myreferences.py>
% \end{pythoncode}
% ^^A ]]] End of subsection `businessdou-skill'.
%
% \subsection{中堅社員研修}^^A [[[
% \begin{pythoncode}
%<*myreferences.py>
add_ref(
    type='book',
    key='中堅社員研修',
    title='中堅社員研修',
    publisher='日本能率協会マネジメントセンター',)
%</myreferences.py>
% \end{pythoncode}
% ^^A ]]] End of subsection `中堅社員研修'.
%
% ^^A ]]] End of section `Business'.
%
%    \begin{macrocode}
%<*reference>
@book{mechanics-of-materials,
    author = "日本機械学会",
    title = "材料力学",
    publisher = "日本機械学会",
    year = 2016 }
%</reference>
%    \end{macrocode}
%
%
%
%
% ^^A ]]] End of part `myreferences.py'.
%
