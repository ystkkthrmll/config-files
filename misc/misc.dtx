%
% ^^A Configuring DocStrip [[[
% \iffalse
%<*driver>
\input mydocstrip
\mygenerate[output-files/misc]{
  analog-clock.desktop,
  bookmarks.home,
  bookmarks.office,
  compose.yaml,
  config.home,
  config.office,
  Dockerfile,
  .gitconfig.home,
  .gitconfig.office,
  .gitignore,
  .gitmessage,
  gnome-terminal.desktop,
  google-chrome.desktop,
  mystyle.css,
  requirements.txt,
  utilities.js,
}
\myrenamefile[output-files/misc]{Dockerfile.tex}{Dockerfile}
\myendbatchfile
\mydriver[
  show-notes,
]
%</driver>
% \fi
% ^^A ]]] End of Configuring DocStrip
%
% \mytitle{Miscellaneous configuration files}
%
% \begin{abstract}^^A [[[
% \begin{myitemize}
% \1 \verb|analog-clock.desktop|, \verb|gnome-terminal.desktop|, \verb|google-chrome.desktop| (autostart)
% \1 \verb|bookmarks.home|, \verb|bookmarks.office|
%   \2 GNOME Filesのsidebarに登録するdirectories
% \1 \verb|.gitconfig.home|, \verb|.gitconfig.office|, \verb|.gitignore|
% \end{myitemize}
% \end{abstract}^^A ]]]
%
% \mytableofcontents
%
% \part{Docker}[updated=]^^A [[[
%
% \section{\texttt{compose.yaml}}[updated=]^^A [[[
% \begin{myitemize}
% \1 \bashinline{docker compose up [-d]}
% \1 References:
%   \2 \href{https://zenn.dev/en2enzo2/articles/78d57ec01ed0c5}{【簡単】JupyterlabをDockerで起動}
%   \2 \href{https://www.idnet.co.jp/column/page_187.html}{DockerでJupyterLabの環境を作ろう}
% \end{myitemize}
%
% \iffalse
%<*compose.yaml>
% \fi
% \begin{yamlcode}
version: "3"

services:
  jupyterlab:
    build:
      context: .
      dockerfile: Dockerfile
    container_name: jupyterlab
    working_dir: /work
    volumes:
    - ${HOME}:/work
    ports:
    - "8888:8888"
    environment:
    - TZ=Asia/Tokyo
    command:
      jupyter-lab --ip 0.0.0.0 --allow-root -b localhost

% \end{yamlcode}
% \iffalse
%</compose.yaml>
% \fi
% ^^A ]]] End of section `compose.yaml'.
%
% \section{\texttt{Dockerfile}}[updated=]^^A [[[
% \begin{myitemize}
% \0 Usage:
% \1 \bashinline{docker build -t jupyterlab .}
% \1 \bashinline{docker run -it --rm --name jupyterlab jupyterlab}
% \0 References:
% \1 \href{https://docs.docker.com/develop/develop-images/dockerfile_best-practices/#apt-get}{apt-get}
% \1 
% \end{myitemize}
%
% \iffalse
%<*Dockerfile>
% \fi
% \begin{dockercode}
FROM python:3
# ENV MISC=/home/yasutaka/Dropbox/configuration-files/miscellaneous/output-files/miscellaneous
RUN apt-get update && apt-get install -y \
    zsh
COPY requirements.txt ./
RUN pip install -r requirements.txt

% \end{dockercode}
% \iffalse
%</Dockerfile>
% \fi
% ^^A ]]] End of section `Dockerfile'.
%
% \section{\texttt{requirements.txt}}[updated=]^^A [[[
% \iffalse
%<*requirements.txt>
% \fi
% \begin{bashcode}
jupyterlab
jupyterlab-vim
# lingam
pandas
pyyaml
# torch
# torchvision
tqdm

% \end{bashcode}
% \iffalse
%</requirements.txt>
% \fi
% ^^A ]]] End of section `requirements.txt'.
%
% ^^A ]]] End of part `Docker'.
%
% \endinput
%
% \part{config}[updated=]^^A [[[
% \begin{myitemize}
% \1 \href{https://www.ssh.com/academy/ssh/config}{SSH config file for OpenSSH client}
% \1 Host
%   \2 Restricts the following declarations to be only for those hosts that match one of the patterns given after the keyword.
%   The pattern is matched against the host name given on the command line.
% \1 HostName
%   \2 Specifies the real host name to log into.
%   This can be used to specify nicknames or abbreviations for hosts.
%   The default is the name given on the command line.
%   Numeric IP addresses are also permitted (both on the command line and in HostName specifications).
% \1 IdentityFile
%   \2 Specifies a file from which the user's identity key is read when using public key authentication.
% \1 Compression
%   \2 Specifies whether to use compression. yes enables compression.
% \end{myitemize}
%
% \section{common}[updated=]^^A [[[
% \iffalse
%<*config.home|config.office>
% \fi
%
% \subsection{github.com}[updated=]^^A [[[
% \begin{bashcode}
Host github.com
    HostName github.com
    User git
    IdentitiesOnly yes
    IdentityFile ~/.ssh/github-kitahara-neut
    Compression yes

% \end{bashcode}
% ^^A ]]] End of subsection `github.com'.
%
% \iffalse
%</config.home|config.office>
% \fi
% ^^A ]]] End of section `common'.
%
% \section{home}[updated=]^^A [[[
% \iffalse
%<*config.home>
% \fi
%
% \subsection{github.com}[updated=]^^A [[[
% \begin{bashcode}
Host github.com
    HostName github.com
    User git
    IdentitiesOnly yes
    IdentityFile ~/.ssh/github-ystkkthrmll
    Compression yes

% \end{bashcode}
% ^^A ]]] End of subsection `github.com'.
%
% \iffalse
%</config.home>
% \fi
% ^^A ]]] End of section `common'.
%
% ^^A ]]] End of part `config'.
%
% \part{\texttt{utilities.js}}^^A [[[
% \iffalse
%<*utilities.js>
% \fi
%
% \section{table}^^A [[[
% \begin{myitemize}
% \1 \href{https://www.w3schools.com/howto/howto_js_filter_table.asp}{How TO - Filter/Search Table}
% \1 \javascriptinline{style.display}
%   \2 \href{https://magazine.techacademy.jp/magazine/33333}{JavaScriptのstyle.displayプロパティの使い方を現役エンジニアが解説【初心者向け】}
% \end{myitemize}
%
% \begin{javascriptcode}
function myFilter() {
  // Declare variables
  var checkbox, filter, table, tables, tr, td, i, txtValue;
  checkbox = document.getElementById("showAll");
  // filter = input.value.toUpperCase();
  tables = document.getElementsByTagName("table");

  for (table of tables) {
    // table = document.getElementByClassName("dataframe");
    tr = table.getElementsByTagName("tr");

    // Loop through all table rows, and hide those who don't match the search query
    for (i = 0; i < tr.length; i++) {
      td = tr[i].getElementsByTagName("td")[1];
      if (td) {
        txtValue = td.textContent || td.innerText;
        if (checkbox.checked == false && txtValue == "completed") {
          tr[i].style.display = "none";
        } else {
          tr[i].style.display = "";
        }
      }
    }
  }
}

function mySearch() {
  // Declare variables
  var input, filter, table, tr, td, i, txtValue;
  input = document.getElementById("myInput");
  filter = input.value.toUpperCase();
  table = document.getElementById("table-to-do list");
  // table = document.getElementByClassName("dataframe");
  tr = table.getElementsByTagName("tr");

  // Loop through all table rows, and hide those who don't match the search query
  for (i = 0; i < tr.length; i++) {
    td = tr[i].getElementsByTagName("td")[1];
    if (td) {
      txtValue = td.textContent || td.innerText;
      if (txtValue.toUpperCase().indexOf(filter) > -1) {
        tr[i].style.display = "";
      } else {
        tr[i].style.display = "none";
      }
    }
  }
}

% \end{javascriptcode}
% ^^A ]]] End of section `table'.
%
% \section{sidebar}^^A [[[
% \begin{javascriptcode}
function openSidebar() {
 document.getElementsByTagName("main")[0].style.marginLeft = "25%";
 document.getElementById("mySidebar").style.width = "25%";
 document.getElementById("mySidebar").style.display = "block";
 document.getElementById("openNav").style.display = 'none';
}

function closeSidebar() {
 document.getElementsByTagName("main")[0].style.marginLeft = "0%";
 document.getElementById("mySidebar").style.display = "none";
 document.getElementById("openNav").style.display = "inline-block";
}

% \end{javascriptcode}
% ^^A ]]] End of section `sidebar'.
%
% \iffalse
%</utilities.js>
% \fi
% ^^A ]]] End of part `utilities.js'.
%
% \part{\texttt{mystyle.css}}^^A [[[
% \iffalse
%<*mystyle.css>
% \fi
%
% \section{Notes}^^A [[[
% \begin{myitemize}
% \1 \cssinline{display}
%   \2 \href{https://saruwakakun.com/html-css/basic/display}{【CSS】displayの使い方を総まとめ!inlineやblockの違いは?}
% \1 \cssinline{box-shadow}
%  \2 \href{https://saruwakakun.com/html-css/basic/box-shadow}{【CSS】box-shadowで影をつける方法とサンプル集}
% \1 \cssinline{overflow}
%   \2 overflowとはCSSのプロパティの1つで「要素のボックスからはみ出た部分をどう扱うか」を指定します。
%   \urlref{https://saruwakakun.com/html-css/basic/overflow}{【CSS】overflowの使い方:hiddenやscrollの違いは?}
% \1 \cssinline{text-transform}
%   \2 text-transformとはテキストにおける英数字の大文字、小文字の表示を指定するプロパティです。
%   \urlref{https://html-coding.co.jp/annex/dictionary/css/text-transform/}{text-transform}
% \1 \cssinline{transition}
%   \2 transition効果(時間的変化)をまとめて指定する
%   \urlref{http://www.htmq.com/css3/transition.shtml}{transition-CSS3リファレンス}
% \end{myitemize}
% ^^A ]]] End of section `Notes'.
%
% \section{header}^^A [[[
% \begin{myitemize}
% \1 \href{https://www.w3schools.com/howto/howto_css_fixed_menu.asp}{How TO - Fixed Menu}
% \end{myitemize}
%
% \begin{csscode}
header {
  overflow: hidden;
  background-color: black;
  position: fixed; /* Set the navbar to fixed position */
  top: 0; /* Position the navbar at the top of the page */
  left: 0;
  width: 100%; /* Full width */
}
/* Links inside the navbar */
header button {
  background-color: black;
  display: block;
  color: #f2f2f2;
  text-align: center;
  padding: 14px 16px;
  text-decoration: none;
}
/* Change background on mouse-over */
header button:hover {
  background: #ddd;
  color: black;
}

header time {
  background: black;
  color: white;
  text-align: right;
}
% \end{csscode}
% ^^A ]]] End of section `header'.
%
% \section{main}^^A [[[
% \begin{csscode}
main {
%  margin-top: 50px; /* Add a top margin to avoid content overlay */
%  top: 50px; /* Position the navbar at the top of the page */
  padding: 10px;
  padding-top: 80px;
  transition: 0.5s;
  font-family: sans-serif;
  font-size: 20px;
}

h1 {
  text-transform: capitalize;
}

% \end{csscode}
% ^^A ]]] End of section `main'.
%
% \section{sidebar}^^A [[[
% \begin{myitemize}
% \1 \href{https://www.w3schools.com/w3css/tryit.asp?filename=tryw3css_sidebar_accordion}{Sidebar Accordion}
% \1 \href{https://www.w3schools.com/w3css/4/w3.css}{w3.css}
% \end{myitemize}
%
% \begin{csscode}
.sidebar {
  transition: 0.5s;
  top: 0;
  left: 0;
  height: 100%;
  background-color: black;
  position: fixed!important;
  z-index: 1;
  overflow: auto;
  box-shadow: 2px 2px 4px gray;
}

.sidebar button {
  display: block;
  font-size: 29px;
  background-color: blue;
  color: white;
  transition: 0.5s;
}

.sidebar a {
  display: block;
  font-size: 25px;
  background-color: black;
  color: white;
  transition: 0.5s;
}

.sidebar a:hover {
  display: block;
  font-size: 35px;
  background-color: white;
  color: black;
  transition: 1.0s;
}

% \end{csscode}
% ^^A ]]] End of section `sidebar'.
%
% \section{table}^^A [[[
% \begin{pythoncode}
table {
%    text-transform: capitalize;
    border-collapse: collapse;
    margin: 25px 0;
    font-size: 20px;
    font-family: sans-serif;
    min-width: 400px;
    box-shadow: 0 0 20px rgba(0, 0, 0, 0.15);
}
thead tr {
    background-color: #009879;
    color: #ffffff;
    text-align: left;
}
th, td {
    padding: 12px 15px;
}
tbody tr {
    border-bottom: 1px solid #dddddd;
}

tbody tr:nth-of-type(even) {
    background-color: #f3f3f3;
}

tbody tr:last-of-type {
    border-bottom: 2px solid #009879;
}
tbody tr.active-row {
    font-weight: bold;
    color: #009879;
}

tr:hover {background-color: #d9efff;}

tr:hover td {background-color: #d9efff;}

% \end{pythoncode}
% ^^A ]]] End of section `table'.
%
% \iffalse
%</mystyle.css>
% \fi
% ^^A ]]] End of part `mystyle.css'.
%
% \part{autostart}[updated=2022-01-09T09:13:10]^^A [[[
%
% \begin{concealableitemize}^^A [[[
% \1 The origin of `autostart'
%   \2 \verb|~/.config/autostart/|
%   \concealablemarginalnote{Updated: 2022-01-03T20:17:31}
% \end{concealableitemize}^^A ]]]
%
% \section{Notes}[updated=2022-01-09T09:08:15]^^A [[[
% \begin{myitemize}
% \1 In your home directory, there's a folder called \verb|.config/autostart|.
% Anything placed in this folder will be run when you login
% \urlref{https://askubuntu.com/questions/178567/how-to-add-a-program-as-a-startup-application-from-a-terminal}{How to add a program as a startup application from a terminal?}
% \1 \href{https://askubuntu.com/questions/598195/how-to-add-a-script-to-startup-applications-from-the-command-line}{How to add a script to Startup Applications from the command line? [duplicate]}
% \1 Filename
%   \2 Use hyphens, not underscores, to separate words
%     \3 \verb|/usr/share/applications/google-chrome.desktop|, \verb|/usr/share/applications/setup-mozc.desktop|, etc.
% \1 Syntax
%   \2 \href{https://wiki.archlinux.org/title/desktop_entries#Application_entry}{Application entry}
% \end{myitemize}
% ^^A ]]] End of section `Notes'.
%
% \section{Implemantation}[updated=2022-01-09T09:10:27]^^A [[[
%
% \subsection{\texttt{analog-clock.desktop}}[updated=2022-01-09T09:09:33]^^A [[[
%
% \iffalse
%<*analog-clock.desktop>
% \fi
%
% \begin{inicode}
% [Desktop Entry]
% Type=Application
% Exec=python3 /home/yasutaka/Dropbox/configuration-files/miscellaneous/output-files/python/analog_clock.py
% \end{inicode}
%
% \iffalse
%</analog-clock.desktop>
% \fi
%
% ^^A ]]] End of subsection `analog-clock.desktop'.
%
% \subsection{\texttt{gnome-terminal.desktop}}[updated=2021-12-19T14:42:45]^^A [[[
% \begin{myitemize}
% \1 \bashinline*{--geometry=...}
%   \2 windowサイズの指定
% \end{myitemize}
%
% \iffalse
%<*gnome-terminal.desktop>
% \fi
%
% \begin{inicode}
[Desktop Entry]
Type=Application
Exec=gnome-terminal --geometry=135x68+0
% \end{inicode}
%
% \iffalse
%</gnome-terminal.desktop>
% \fi
%
% ^^A ]]] End of subsection `gnome-terminal.desktop'.
%
% \subsection{\texttt{google-chrome.desktop}}[updated=2021-12-19T14:36:16]^^A [[[
%
% \iffalse
%<*google-chrome.desktop>
% \fi
%
% \begin{inicode}
[Desktop Entry]
Type=Application
Exec=google-chrome
% \end{inicode}
%
% \iffalse
%</google-chrome.desktop>
% \fi
%
% ^^A ]]] End of subsection `google-chrome.desktop'.
%
% ^^A ]]] End of section `Implemantation'.
%
% ^^A ]]] End of part `autostart'.
%
% \part{\texttt{bookmarks}}[updated=2022-01-09T09:13:19]^^A [[[
%
% \section{Notes}[updated=2022-01-03T20:27:04]^^A [[[
% \begin{myitemize}
% \1 The origin of `GNOME Files'
%   \2 \href{https://wiki.archlinux.org/title/GNOME/Files}{GNOME/Files}
% \1 \verb|Ctrl+D| add to sidebar
% \urlref{https://askubuntu.com/questions/83118/create-a-link-to-a-folder-on-the-left-panel-of-nautilus-file-manager}{Create a link to a folder on the left panel of Nautilus file manager?}
% \1 Remove folders from the places sidebar
% \urlref{https://wiki.archlinux.org/index.php/GNOME/Files#Remove_folders_from_the_places_sidebar}{Remove folders from the places sidebar}
% \end{myitemize}
% ^^A ]]] End of section `Notes'.
%
% \section{Implemantation}[updated=2022-01-09T09:12:52]^^A [[[
%
% \subsection{Writing common directories}[updated=2022-01-09T09:12:02]^^A [[[
%
% \iffalse
%<*bookmarks.home|bookmarks.office>
% \fi
%
% \begin{bashcode}
file:///home/yasutaka/Dropbox/configuration-files configuration-files
file:///home/yasutaka/Dropbox/my-documents-pdf my-documents-pdf
% \end{bashcode}
%
% \iffalse
%</bookmarks.home|bookmarks.office>
% \fi
%
% ^^A ]]] End of subsection `Writing common directories'.
%
% \subsection{Writing \texttt{bookmarks.home}}[updated=2021-12-18T20:39:54]^^A [[[
%
% \iffalse
%<*bookmarks.home>
% \fi
%
% \begin{bashcode}
file:///home/yasutaka/my-documents/computer computer
file:///home/yasutaka/my-documents/english english
file:///home/yasutaka/my-documents/job job
file:///home/yasutaka/my-documents/math math
file:///home/yasutaka/my-documents/miscellaneous miscellaneous
file:///home/yasutaka/my-documents/physics physics
% \end{bashcode}
%
% \iffalse
%</bookmarks.home>
% \fi
%
% ^^A ]]] End of subsection `Writing bookmarks.home'.
%
% \subsection{Writing \texttt{bookmarks.office}}[updated=2021-12-18T20:41:09]^^A [[[
%
% \iffalse
%<*bookmarks.office>
% \fi
%
% \begin{bashcode}
file:///mnt/hgfs/job job
% \end{bashcode}
%
% \iffalse
%</bookmarks.office>
% \fi
%
% ^^A ]]] End of subsection `Writing bookmarks.office'.
%
% ^^A ]]] End of section `Implemantation'.
%
% ^^A ]]] End of part `bookmarks'.
%
% \part{Git}^^A [[[
%
% \section{Notes}^^A [[[
% \begin{myitemize}
% \1 About \verb|.gitconfig|
%   \2 Git's configuration files are all simple text files in the style of \verb|.ini| files.
%   They record various choices and settings used by many Git commands
%   \mycite[page=28]{version-control-with-git}.
%   \2 \verb|~/.gitconfig|
%   User-specific configuration settings manipulated with the \verb|--global| option
%   \mycite[page=28]{version-control-with-git}.
%   \2 \verb|~/.gitconfig| or \verb|~/.config/git/config| file:
%   Values specific personally to you, the user.
%   You can make Git read and write to this file specifically by passing the \verb|--global| option,
%   and this affects \emph{all} of the repositories you work with on your system
%   \urlref{https://git-scm.com/book/en/v2/Getting-Started-First-Time-Git-Setup}{1.6 Getting Started - First-Time Git Setup}.
% \1 About \verb|.gitignore|
%   \2 You can also create a global \verb|.gitignore| file,
%   which is a list of rules for ignoring files in \myemph{every} Git repository on your computer.
%   For example, you might create the file at \verb|~/.gitignore_global| and add some rules to it
%   \urlref{https://help.github.com/en/github/using-git/ignoring-files#create-a-global-gitignore}{Create a global .gitignore}.
%   \2 \verb|.gitignore_global|とは,
%   Gitで管理する全てのフォルダに適用される\verb|.gitignore|のようなファイルです.
%   このファイルを作成して,常に無視したいファイルを追記することでGitの管理下から除外できます%
%   \urlref{https://qiita.com/pira/items/dd67077c5b414c8eb59d#gitignore_global%E3%81%A8%E3%81%AF}{gitignoreglobalを作成する on OSX}.
% \1
% \end{myitemize}
% ^^A ]]] End of section `Notes'.
%
% \section{Implementation}^^A [[[
%
% \subsection{\texttt{.gitconfig}}^^A [[[
%
% \subsubsection{user}[updated=2022-01-04T10:41:28]^^A [[[
% \begin{myitemize}
% \1 \href{https://stackoverflow.com/questions/14754762/can-gitconfig-options-be-set-conditionally}{Can gitconfig options be set conditionally?}にあるような
% \iniinline{email = "if ..."}や\iniinline{email = "!if ..."}では機能しなかった
% \end{myitemize}
%
% \begin{inicode}
%<*.gitconfig.home>
[user]
    name = Yasutaka Kitahara @ home
    email = 0fh4922388417x@gmail.com
%</.gitconfig.home>
%<*.gitconfig.office>
[user]
    name = Yasutaka Kitahara @ office
    email = kitahara.yasutaka@neut.co.jp
%</.gitconfig.office>
% \end{inicode}
% ^^A ]]] End of subsubsection `user'.
%
% \iffalse
%<*.gitconfig.home|.gitconfig.office>
% \fi
%
% \subsubsection{alias}^^A [[[
% \begin{myitemize}
% \1 References
%   \2 \href{https://git-scm.com/book/en/v2/Git-Basics-Git-Aliases}{2.7 Git Basics - Git Aliases}
% \end{myitemize}
%
% \begin{inicode}
[alias]
    br = branch
    ci = commit
    co = checkout
    st = status
% \end{inicode}
% ^^A ]]] End of subsubsection `alias'.
%
% \subsubsection{color}^^A [[[
% \begin{myitemize}
% \1 \href{https://unix.stackexchange.com/questions/44266/how-to-colorize-output-of-git}{How to colorize output of git?}
% \end{myitemize}
%
% \begin{inicode}
[color "status"]
    added = cyan
    changed = green bold
    untracked = red bold
% \end{inicode}
% ^^A ]]] End of subsubsection `color'.
%
% \subsubsection{commit}^^A [[[
% \begin{myitemize}
% \1 \href{https://gist.github.com/lisawolderiksen/a7b99d94c92c6671181611be1641c733#template-file}{Using Git Commit Message Templates to Write Better Commit Messages}
% \end{myitemize}
%
% \begin{inicode}
[commit]
    template = ~/Dropbox/config-files/misc/output-files/misc/.gitmessage
% \end{inicode}
% ^^A ]]] End of subsubsection `commit'.
%
% \subsubsection{core}^^A [[[
% \begin{myitemize}
% \1 \verb|$ git config --global core.excludesfile ~/.gitignore_global|
%   \2 \href{https://git-scm.com/docs/gitignore#_description#:~:text=Patterns%20which%20a%20user%20wants%20Git%20to%20ignore%20in%20all%20situations}{Git - gitignore Documentation}
%   \2 \href{https://help.github.com/en/github/using-git/ignoring-files#create-a-global-gitignore0}{参照先}
% \1 \verb|$ git config --global core.editor nvim| (cf. "Pro Git")
% \1
% \end{myitemize}
%
% \begin{inicode}
[core]
    excludesfile = ~/Dropbox/config-files/misc/output-files/misc/.gitignore
    editor = nvim
% \end{inicode}
% ^^A ]]] End of subsubsection `core'.
%
% \iffalse
%</.gitconfig.home|.gitconfig.office>
% \fi
%
% ^^A ]]] End of subsection `.gitconfig'.
%
% \subsection{\texttt{.gitignore}}[updated=2022-01-09T09:35:38]^^A [[[
% \begin{myitemize}
% \1 \verb|.gitignore|ファイルの書き方は\mycite[page=60]{version-control-with-git}
% \1 各拡張子のメモ
%   \2 \verb|.bib|: \LaTeX 参考文献ファイル
%   \2 \verb|.csv|: Comma-separated values file (CSV)
%   \2 \verb|.html|: HTML file
%   \2 \verb|.jpg|: image file
%   \2 \verb|.log|: log file
%   \2 \verb|.pdf|: Portable Document Format (PDF)
%   \2 \verb|.png|: image file
%   \2 \verb|.pyc|: This is the compiled bytecode.
%   If you import a module, python will build a \verb|*.pyc| file that contains the bytecode
%   to make importing it again later easier (and faster).
%   \urlref{https://stackoverflow.com/questions/8822335/what-do-the-python-file-extensions-pyc-pyd-pyo-stand-for}{What do the python file extensions, .pyc .pyd .pyo stand for?}
%   \2 \verb|.zip|: 圧縮ファイル
%   \2 \verb|.ipynb_checkpoints|: jupyter notebookで生成されるファイル
%   \2 \verb|output-files|: mydocstrip.texで出力されるファイルの出力先
% \end{myitemize}
%
% \iffalse
%<*.gitignore>
% \fi
% \begin{bashcode}
*.bib
*.csv
*.html
*.jpg
*.log
*.pdf
*.png
*.pyc
*.xlsx
*.zip
.ipynb_checkpoints
output-files
% \end{bashcode}
% \iffalse
%</.gitignore>
% \fi
% ^^A ]]] End of subsection `.gitignore'.
%
% \subsection{\texttt{.gitmessage}}[updated=]^^A [[[
% \iffalse
%<*.gitmessage>
% \fi
% \begin{bashcode}
# ===========================
# feat: Initialize Project
#
# * Add README.md
# * Generate Rails scaffolding
# * Add .gitignore
# ===========================
% \end{bashcode}
% \iffalse
%</.gitmessage>
% \fi
% ^^A ]]] End of subsection `.gitmessage'.
%
% ^^A ]]] End of section `Implementation'.
%
% ^^A ]]] End of part `Git'.
%
% ^^A End of file `miscellaneous.dtx'.
